\documentclass[]{article}
\usepackage{lmodern}
\usepackage{amssymb,amsmath}
\usepackage{ifxetex,ifluatex}
\usepackage{fixltx2e} % provides \textsubscript
\ifnum 0\ifxetex 1\fi\ifluatex 1\fi=0 % if pdftex
  \usepackage[T1]{fontenc}
  \usepackage[utf8]{inputenc}
\else % if luatex or xelatex
  \ifxetex
    \usepackage{mathspec}
  \else
    \usepackage{fontspec}
  \fi
  \defaultfontfeatures{Ligatures=TeX,Scale=MatchLowercase}
\fi
% use upquote if available, for straight quotes in verbatim environments
\IfFileExists{upquote.sty}{\usepackage{upquote}}{}
% use microtype if available
\IfFileExists{microtype.sty}{%
\usepackage{microtype}
\UseMicrotypeSet[protrusion]{basicmath} % disable protrusion for tt fonts
}{}
\usepackage[margin=1in]{geometry}
\usepackage{hyperref}
\hypersetup{unicode=true,
            pdftitle={Estimation of Contract Schedules using Constrained B-Splines},
            pdfauthor={Michael Chirico},
            pdfborder={0 0 0},
            breaklinks=true}
\urlstyle{same}  % don't use monospace font for urls
\usepackage{graphicx,grffile}
\makeatletter
\def\maxwidth{\ifdim\Gin@nat@width>\linewidth\linewidth\else\Gin@nat@width\fi}
\def\maxheight{\ifdim\Gin@nat@height>\textheight\textheight\else\Gin@nat@height\fi}
\makeatother
% Scale images if necessary, so that they will not overflow the page
% margins by default, and it is still possible to overwrite the defaults
% using explicit options in \includegraphics[width, height, ...]{}
\setkeys{Gin}{width=\maxwidth,height=\maxheight,keepaspectratio}
\IfFileExists{parskip.sty}{%
\usepackage{parskip}
}{% else
\setlength{\parindent}{0pt}
\setlength{\parskip}{6pt plus 2pt minus 1pt}
}
\setlength{\emergencystretch}{3em}  % prevent overfull lines
\providecommand{\tightlist}{%
  \setlength{\itemsep}{0pt}\setlength{\parskip}{0pt}}
\setcounter{secnumdepth}{0}
% Redefines (sub)paragraphs to behave more like sections
\ifx\paragraph\undefined\else
\let\oldparagraph\paragraph
\renewcommand{\paragraph}[1]{\oldparagraph{#1}\mbox{}}
\fi
\ifx\subparagraph\undefined\else
\let\oldsubparagraph\subparagraph
\renewcommand{\subparagraph}[1]{\oldsubparagraph{#1}\mbox{}}
\fi

%%% Use protect on footnotes to avoid problems with footnotes in titles
\let\rmarkdownfootnote\footnote%
\def\footnote{\protect\rmarkdownfootnote}

%%% Change title format to be more compact
\usepackage{titling}

% Create subtitle command for use in maketitle
\newcommand{\subtitle}[1]{
  \posttitle{
    \begin{center}\large#1\end{center}
    }
}

\setlength{\droptitle}{-2em}
  \title{Estimation of Contract Schedules using Constrained B-Splines}
  \pretitle{\vspace{\droptitle}\centering\huge}
  \posttitle{\par}
  \author{Michael Chirico}
  \preauthor{\centering\large\emph}
  \postauthor{\par}
  \predate{\centering\large\emph}
  \postdate{\par}
  \date{February 18, 2017}

\usepackage{theapa}

\begin{document}
\maketitle
\begin{abstract}
Most unionized teachers are paid according to a salary schedule
(specifying wages as an increasing function of tenure and certification)
explicated in contracts collectively bargained at the district level.
With this in hand, teachers are able to infer their future potential
wage trajectories at their own and other potential district employers.
Lacking the physical contract faced by the teachers, an econometrician
armed only with administrative data reporting actual wages in a given
year must use some imputation techniques to deduce the wage structure.
We explore the utility of natural Constrained B-Splines (COBS) to this
end. COBS are an enhanced version of the traditional semiparametric
splines technique enhanced by the ability to impose a monotonicity
constraint on the resultant curve.
\end{abstract}

\section{Introduction}\label{introduction}

For many years, the ubiquitous characteristic of collectively bargained
teachers' contracts has been the salary table, which gives a mapping
from the calendar year, a teacher's experience (their length of tenure
at the current district), and their certification (typically Master's
vs.~Bachelor's degree) to their wage. This table gives current teachers
a clear understanding of how their pay will advance as a function of
their labor inputs, and thereby gives forward-looking potential teachers
and potential migrant teachers a clear understanding of their potential
pay arcs under a district-switching decision-making framework,
especially given that this information is typically openly available.

It would behoove an econometrician seeking to understand education labor
market dynamics, then, to incorporate this information on future pay
into their statistical modeling framework. Unfortunately, this data is
typically not available in a format lending itself to easy analysis at
scale -- whether locked inside idiosyncratically formatted and
sporadically-available contract PDFs or hidden behind large-scale
freedom of information act inquiries, the temporal and financial costs
of scraping such data into a usable form can be substantial.

Much more common in empirical settings is access to teacher-year-level
salary data of the form
\(y_{i, t} = y(\tau_{i, t}, c_{i, t}, d_{i, t}) + \varepsilon_{i, t}\),
where \(\tau_{i,t}\), \(c_{i, t}\) and \(d_{i, t}\) are the tenure,
certification, and district of teacher \(i\) in year \(t\), and
\(\varepsilon_{i, t}\) represents unaccount factors affecting the wage
(e.g., not all teachers work full time, and many teachers supplement
their income with additional duties like coaching). The goal of this
paper is to give one approach and some empirical lessons for trying to
estimate the underlying mapping \(y(\tau, c, d)\) from such data.

\section{Method}\label{method}

There are a multitude of inference/imputation techniques suitable to the
inference of a latent function of unknown parametric form available in
the statistician/econometrician's palette. The powerful flexibility of
nonparametric approaches (local regression, splines, Random Fourier
Feature expansions) is a double-edged sword; as it happens, in this
particular setting, even if we know linearity is not a reasonable
functional form restriction, we do know some very basic properties of
the underlying tenure-wage curves that will be violated in general by
uninformed estimation techniques. In particular, we know that such
tenure-wage curves are non-decreasing and that they are non-negative,
i.e., \(y(\tau', c, d) >= y(\tau, c, d)\) whenever \(\tau' >= \tau\),
and \(y(0, c, d) \geq 0\).

He and Ng (\protect\hyperlink{ref-he}{1999}) introduce a linear
programming approach to incorporating monotonicity and curvature
restraints to quantile regression spline estimation techniques, and Ng
and Maechler (\protect\hyperlink{ref-ng}{2007}) present an overview of
the R package \texttt{cobs} which gives an efficient implementation of
this approach (COBS standing for Constrained B-Splines). The basic idea
of quantile regression spline estimation is to swap out the standard
squared loss function for a quantile-dependent weighted absolute loss
function to target conditional quantiles instead of conditional means.
Monotonicity, point, and curvature restrictions enter as penalized terms
to the objective function; \texttt{cobs} expresses this in a fashion
which facilitates the application of standard linear programming
techniques for efficiency, and handles internally the issues of knot
selection and penalty parameter assignment through cross-validation.

\section{Data}\label{data}

The state of Wisconsin's Department of Public Instruction (DPI) releases
annual Salary, Position \& Demographic reports through the WISEstaff
data collection system, and these reports represent ``a point-in-time
collection of all staff members in public schools as of the 3rd Friday
of September\ldots{}'' (DPI \protect\hyperlink{ref-dpi}{2017}).

\section{References}\label{references}

\bibliographystyle{theapa}\bibliography{references}

\hypertarget{refs}{}
\hypertarget{ref-dpi}{}
DPI. 2017. ``School Staff: Salary, Position \& Demographic Reports.''
\url{https://dpi.wi.gov/cst/data-collections/staff/published-data}.

\hypertarget{ref-he}{}
He, Xuming, and Pin Ng. 1999. ``COBS: Qualitatively Constrained
Smoothing via Linear Programming.'' \emph{Computational Statistics} 14
(3). Heidelberg: Physica-Verlag,{[}1992-: 315--38.

\hypertarget{ref-ng}{}
Ng, Pin, and Martin Maechler. 2007. ``A Fast and Efficient
Implementation of Qualitatively Constrained Quantile Smoothing
Splines.'' \emph{Statistical Modelling} 7 (4). Sage Publications India
Pvt. Ltd, B-42, Panchsheel Enclave, New Delhi: 315--28.


\end{document}
