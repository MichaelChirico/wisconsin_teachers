\documentclass[]{article}
\usepackage{lmodern}
\usepackage{amssymb,amsmath}
\usepackage{ifxetex,ifluatex}
\usepackage{fixltx2e} % provides \textsubscript
\ifnum 0\ifxetex 1\fi\ifluatex 1\fi=0 % if pdftex
  \usepackage[T1]{fontenc}
  \usepackage[utf8]{inputenc}
\else % if luatex or xelatex
  \ifxetex
    \usepackage{mathspec}
  \else
    \usepackage{fontspec}
  \fi
  \defaultfontfeatures{Ligatures=TeX,Scale=MatchLowercase}
\fi
% use upquote if available, for straight quotes in verbatim environments
\IfFileExists{upquote.sty}{\usepackage{upquote}}{}
% use microtype if available
\IfFileExists{microtype.sty}{%
\usepackage{microtype}
\UseMicrotypeSet[protrusion]{basicmath} % disable protrusion for tt fonts
}{}
\usepackage[margin=1in]{geometry}
\usepackage{hyperref}
\hypersetup{unicode=true,
            pdftitle={Estimation of Contract Schedules using Constrained B-Splines},
            pdfauthor={Michael Chirico},
            pdfborder={0 0 0},
            breaklinks=true}
\urlstyle{same}  % don't use monospace font for urls
\usepackage{graphicx,grffile}
\makeatletter
\def\maxwidth{\ifdim\Gin@nat@width>\linewidth\linewidth\else\Gin@nat@width\fi}
\def\maxheight{\ifdim\Gin@nat@height>\textheight\textheight\else\Gin@nat@height\fi}
\makeatother
% Scale images if necessary, so that they will not overflow the page
% margins by default, and it is still possible to overwrite the defaults
% using explicit options in \includegraphics[width, height, ...]{}
\setkeys{Gin}{width=\maxwidth,height=\maxheight,keepaspectratio}
\IfFileExists{parskip.sty}{%
\usepackage{parskip}
}{% else
\setlength{\parindent}{0pt}
\setlength{\parskip}{6pt plus 2pt minus 1pt}
}
\setlength{\emergencystretch}{3em}  % prevent overfull lines
\providecommand{\tightlist}{%
  \setlength{\itemsep}{0pt}\setlength{\parskip}{0pt}}
\setcounter{secnumdepth}{0}
% Redefines (sub)paragraphs to behave more like sections
\ifx\paragraph\undefined\else
\let\oldparagraph\paragraph
\renewcommand{\paragraph}[1]{\oldparagraph{#1}\mbox{}}
\fi
\ifx\subparagraph\undefined\else
\let\oldsubparagraph\subparagraph
\renewcommand{\subparagraph}[1]{\oldsubparagraph{#1}\mbox{}}
\fi

%%% Use protect on footnotes to avoid problems with footnotes in titles
\let\rmarkdownfootnote\footnote%
\def\footnote{\protect\rmarkdownfootnote}

%%% Change title format to be more compact
\usepackage{titling}

% Create subtitle command for use in maketitle
\newcommand{\subtitle}[1]{
  \posttitle{
    \begin{center}\large#1\end{center}
    }
}

\setlength{\droptitle}{-2em}
  \title{Estimation of Contract Schedules using Constrained B-Splines}
  \pretitle{\vspace{\droptitle}\centering\huge}
  \posttitle{\par}
  \author{Michael Chirico}
  \preauthor{\centering\large\emph}
  \postauthor{\par}
  \predate{\centering\large\emph}
  \postdate{\par}
  \date{February 18, 2017}

\usepackage{theapa}

\begin{document}
\maketitle
\begin{abstract}
Most unionized teachers are paid according to a salary schedule
(specifying wages as an increasing function of tenure and certification)
explicated in contracts collectively bargained at the district level.
With this in hand, teachers are able to infer their future potential
wage trajectories at their own and other potential district employers.
Lacking the physical contract faced by the teachers, an econometrician
armed only with administrative data reporting actual wages in a given
year must use some imputation techniques to deduce the wage structure.
We explore the utility of natural Constrained B-Splines (COBS) to this
end. COBS are an enhanced version of the traditional semiparametric
splines technique enhanced by the ability to impose a monotonicity
constraint on the resultant curve.
\end{abstract}

\section{Introduction}\label{introduction}

For many years, the ubiquitous characteristic of collectively bargained
teachers' contracts has been the salary table, which gives a mapping
from the calendar year, a teacher's experience (their length of tenure
at the current district), and their certification (typically Master's
vs.~Bachelor's degree) to their wage. This table gives current teachers
a clear understanding of how their pay will advance as a function of
their labor inputs, and thereby gives forward-looking potential teachers
and potential migrant teachers a clear understanding of their potential
pay arcs under a district-switching decision-making framework,
especially given that this information is typically openly available.

It would behoove an econometrician seeking to understand education labor
market dynamics, then, to incorporate this information on future pay
into their statistical modeling framework. Unfortunately, this data is
typically not available in a format lending itself to easy analysis at
scale -- whether locked inside idiosyncratically formatted and
sporadically-available contract PDFs or hidden behind large-scale
freedom of information act inquiries, the temporal and financial costs
of scraping such data into a usable form can be substantial.

Much more common in empirical settings is access to teacher-year-level
salary data of the form
\(y_{i, t} = y(\tau_{i, t}, c_{i, t}, d_{i, t}) + \varepsilon_{i, t}\),
where \(\tau_{i,t}\), \(c_{i, t}\) and \(d_{i, t}\) are the tenure,
certification, and district of teacher \(i\) in year \(t\), and
\(\varepsilon_{i, t}\) represents unaccount factors affecting the wage
(e.g., not all teachers work full time, and many teachers supplement
their income with additional duties like coaching). The goal of this
paper is to give one approach and some empirical lessons for trying to
estimate the underlying mapping \(y(\tau, c, d)\) from such data.

\section{Method}\label{method}

There are a multitude of inference/imputation techniques suitable to the
inference of a latent function of unknown parametric form available in
the statistician/econometrician's palette. The powerful flexibility of
nonparametric approaches (local regression, splines, Random Fourier
Feature expansions) is a double-edged sword; as it happens, in this
particular setting, even if we know linearity is not a reasonable
functional form restriction, we do know some very basic properties of
the underlying tenure-wage curves that will be violated in general by
uninformed estimation techniques. In particular, we know that such
tenure-wage curves are non-decreasing and that they are non-negative,
i.e., \(y(\tau', c, d) >= y(\tau, c, d)\) whenever \(\tau' >= \tau\),
and \(y(0, c, d) \geq 0\).

He and Ng (\protect\hyperlink{ref-he}{1999}) introduce a linear
programming approach to incorporating monotonicity and curvature
restraints to quantile regression spline estimation techniques, and Ng
and Maechler (\protect\hyperlink{ref-ng}{2007}) present an overview of
the R package \texttt{cobs} which gives an efficient implementation of
this approach (COBS standing for Constrained B-Splines). The basic idea
of quantile regression spline estimation is to swap out the standard
squared loss function for a quantile-dependent weighted absolute loss
function to target conditional quantiles instead of conditional means.
Monotonicity, point, and curvature restrictions enter as penalized terms
to the objective function; \texttt{cobs} expresses this in a fashion
which facilitates the application of standard linear programming
techniques for efficiency, and handles internally the issues of knot
selection and penalty parameter assignment through cross-validation.

\section{Data}\label{data}

The state of Wisconsin's Department of Public Instruction (DPI) releases
annual Salary, Position \& Demographic reports through the WISEstaff
data collection system, and these reports represent ``a point-in-time
collection of all staff members in public schools as of the 3rd Friday
of September\ldots{}'' (DPI \protect\hyperlink{ref-dpi}{2017}). The
crucial data elements from each report are the Highest Degree Code
(certification), Total Experience in Education (tenure), Total Salary,
Total Fringe/Employee Benefits, and Agency of Work Location Code (unique
identifier of district). We pull data from the 1994-95 academic year
(AY) through AY2015-16 for a total of 3,588,614 teacher-position-year
observations (many teachers serve in multiple roles within a
school/district, and each of these is filed separately in the DPI
system).

Data are first fed through the matching algorithm described in further
detail in a companion paper. As a consequence, 26,302 (0.7\%)
observations are lost on account of belonging to teachers who could not
be uniquely identified in a given year of data due to sharing a first
name, last name, and birth year with another teacher in the data.
Specific to the exercise at hand, with data reliability and precision in
mind, we make a series of further restrictions on the data, as depicted
in Figure \ref{fig:samp_res}.

To wit, from the full staff file are first eliminated all non-teaching
positions; most common among the dropped observations are ``Other
Support Staff'', ``Program Aide'', and ``Short-term Substitute
Teacher,'' all of whom are likely to have their pay structured in ways
that differ substantially from teachers (e.g., these positions are often
part of a separately-bargained contract). We then remove all teachers
whose highest degree is neither a Bachelor's nor a Master's degree, a
minor restriction which allows substantial clarity in understanding
districts' pay structure. For more precision, we next eliminate teachers
not categorized as professionals in regular education (as opposed by and
large to those in special education)\footnote{Sensitivity analyses to
  this and other restrictions are given in the Appendix.}.

The next two restrictions are related to a teacher's full-time
equivalency. Full-time equivalency is for the most part used to break
down the relative use of staff's time on various duties (for example,
non-teaching time), or to place substitute teachers on a pay grade
commensurate with their work load. Given the possibility of split duties
to dilute the contribution of a teacher's classroom role to their pay,
we eliminate teachers whose full-time equivalency across teaching roles
is not 100; we then select only the highest-intensity (highest-FTE) role
for each teacher in a given year\footnote{With a very small number of
  likely-erroneous exceptions, each teacher's observation at a given
  district in a given year is associated with exactly one value for
  their pay (so that pay is not broken down by position), which means
  that neither can we ``scale up'' less-than-full-time teachers' pay to
  full time values, nor is it particularly important to select the
  highest-intensity role for each teacher.}.

The next three restrictions have only a minor effect on sample size, and
they are to eliminate districts not categorized as Wisconsin Public
Schools (for the most part, observations failing this test are teachers
employed by CESAs, Wisconsin's supra-district administrative unit), to
eliminate teachers recorded as having worked a period of time different
from a full school year, and to eliminate a small number of teachers
whose pay was implausibly small for a full time staff member (less than
\$10,000 in a year). After this, we cut teachers outside 1-30 years'
experience for clarity and precision (sample sizes begin to drop off
around 25 years of teaching), after which we eliminate teachers assigned
to non-teaching subjects (health, academic support, gifted \& talented
support, etc.).

Finally, we make a series of cuts which are either required for COBS to
function, or else increase the reliability of its output substantially.
The most noteworthy/far-reaching of these numerical restrictions is to
eliminate any teachers working in districts where there are not at least
20 total teachers in each degree track. While ultimately arbitrary, this
number is reasonable to limit the potential effect of an individual
teacher given the function to be fit to 30 levels of experience with
minimal functional form restrictions. The other numerical flags require
both the BA \& MA track to be represented at a district, for at least 7
different levels of experience to be represented within a degree track,
and for at least 5 unique values of the two measures of pay (salary and
fringe benefits) to be available in each degree track; any teacher at a
district failing these tests is dropped.

The sum total of all of these restrictions leaves us with an analysis
sample of 689,678 teacher-year observations, made up of 94,370
individual teachers in 242 districts over 22 years.

\begin{figure}[htbp]
\centering
\includegraphics{cobs_paper_files/figure-latex/drop_plot-1.pdf}
\caption{\label{fig:samp_res}Sample Restrictions}
\end{figure}

\includegraphics{cobs_paper_files/figure-latex/overall_pay-1.pdf}

\section{References}\label{references}

\bibliographystyle{theapa}\bibliography{references}

\hypertarget{refs}{}
\hypertarget{ref-dpi}{}
DPI. 2017. ``School Staff: Salary, Position \& Demographic Reports.''
\url{https://dpi.wi.gov/cst/data-collections/staff/published-data}.

\hypertarget{ref-he}{}
He, Xuming, and Pin Ng. 1999. ``COBS: Qualitatively Constrained
Smoothing via Linear Programming.'' \emph{Computational Statistics} 14
(3). Heidelberg: Physica-Verlag,{[}1992-: 315--38.

\hypertarget{ref-ng}{}
Ng, Pin, and Martin Maechler. 2007. ``A Fast and Efficient
Implementation of Qualitatively Constrained Quantile Smoothing
Splines.'' \emph{Statistical Modelling} 7 (4). Sage Publications India
Pvt. Ltd, B-42, Panchsheel Enclave, New Delhi: 315--28.


\end{document}
