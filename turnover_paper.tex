\documentclass[12pt,]{article}
\usepackage{lmodern}
\usepackage{setspace}
\setstretch{1.5}
\usepackage{amssymb,amsmath}
\usepackage{ifxetex,ifluatex}
\usepackage{fixltx2e} % provides \textsubscript
\ifnum 0\ifxetex 1\fi\ifluatex 1\fi=0 % if pdftex
  \usepackage[T1]{fontenc}
  \usepackage[utf8]{inputenc}
\else % if luatex or xelatex
  \ifxetex
    \usepackage{mathspec}
  \else
    \usepackage{fontspec}
  \fi
  \defaultfontfeatures{Ligatures=TeX,Scale=MatchLowercase}
\fi
% use upquote if available, for straight quotes in verbatim environments
\IfFileExists{upquote.sty}{\usepackage{upquote}}{}
% use microtype if available
\IfFileExists{microtype.sty}{%
\usepackage{microtype}
\UseMicrotypeSet[protrusion]{basicmath} % disable protrusion for tt fonts
}{}
\usepackage[margin=1in]{geometry}
\usepackage{hyperref}
\hypersetup{unicode=true,
            pdftitle={Teacher Turnover in Wisconsin},
            pdfauthor={Michael Chirico},
            pdfborder={0 0 0},
            breaklinks=true}
\urlstyle{same}  % don't use monospace font for urls
\usepackage{graphicx,grffile}
\makeatletter
\def\maxwidth{\ifdim\Gin@nat@width>\linewidth\linewidth\else\Gin@nat@width\fi}
\def\maxheight{\ifdim\Gin@nat@height>\textheight\textheight\else\Gin@nat@height\fi}
\makeatother
% Scale images if necessary, so that they will not overflow the page
% margins by default, and it is still possible to overwrite the defaults
% using explicit options in \includegraphics[width, height, ...]{}
\setkeys{Gin}{width=\maxwidth,height=\maxheight,keepaspectratio}
\IfFileExists{parskip.sty}{%
\usepackage{parskip}
}{% else
\setlength{\parindent}{0pt}
\setlength{\parskip}{6pt plus 2pt minus 1pt}
}
\setlength{\emergencystretch}{3em}  % prevent overfull lines
\providecommand{\tightlist}{%
  \setlength{\itemsep}{0pt}\setlength{\parskip}{0pt}}
\setcounter{secnumdepth}{0}
% Redefines (sub)paragraphs to behave more like sections
\ifx\paragraph\undefined\else
\let\oldparagraph\paragraph
\renewcommand{\paragraph}[1]{\oldparagraph{#1}\mbox{}}
\fi
\ifx\subparagraph\undefined\else
\let\oldsubparagraph\subparagraph
\renewcommand{\subparagraph}[1]{\oldsubparagraph{#1}\mbox{}}
\fi

%%% Use protect on footnotes to avoid problems with footnotes in titles
\let\rmarkdownfootnote\footnote%
\def\footnote{\protect\rmarkdownfootnote}

%%% Change title format to be more compact
\usepackage{titling}

% Create subtitle command for use in maketitle
\newcommand{\subtitle}[1]{
  \posttitle{
    \begin{center}\large#1\end{center}
    }
}

\setlength{\droptitle}{-2em}
  \title{Teacher Turnover in Wisconsin}
  \pretitle{\vspace{\droptitle}\centering\huge}
  \posttitle{\par}
  \author{Michael Chirico}
  \preauthor{\centering\large\emph}
  \postauthor{\par}
  \predate{\centering\large\emph}
  \postdate{\par}
  \date{May 13, 2017}

\usepackage{theapa}
\usepackage{array}
\usepackage{multirow}
\newcolumntype{R}[1]{>{\raggedleft\arraybackslash}p{#1\linewidth}}
\usepackage{rotating}
\usepackage{verbatim}
\newcommand{\TAG}[1]{}

\begin{document}
\maketitle
\begin{abstract}
Given the consistently-affirmed importance of teacher quality to student
success, understanding teacher churn is crucial to formulating and
evaluating teacher labor market policy. This paper replicates and
extends the analysis of Hanushek, Kain, and Rivkin
(\protect\hyperlink{ref-hanushek}{2004}) over a longer and more recent
time period in Wisconsin and confirms all of its major findings, namely
that while inter-district pay differentials are a significant
determinant of turnover and that school quality measures are much better
predictors of all three types of churn -- within and between school
districts and out of local public schools. We further implement and
validate a new technique for inferring contracted wage schedules from
observed cross-sectional wage data.
\end{abstract}

\section{Introduction}\label{introduction}

\TAG{BEGIN_INTRO}

Good teachers have large impacts on student achievement\footnote{See,
  e.g., Rockoff (\protect\hyperlink{ref-rockoff}{2004}).}. It is
therefore imperative for public schools to be able to attract and retain
high-quality teachers. Of preeminent concern for policymakers, then, is
the strength of the various manipulable levers at their disposal for
influencing teacher labor markets. More specifically, state and local
education administrators would like to identify the policy implications
of various tools on three types of teacher mobility: intra-district
switching, where due to the collectively bargained nature of most
teachers' salaries, only nonpecuniary considerations matter,
inter-district switching, where teachers move to another school district
in the same state, and exo-district switching, where teachers leave the
public teaching workforce entirely\footnote{Policies that affect the
  supply and quality of new teachers to the profession may also be of
  considerable importance to replenishing and improving the stock of
  teachers over time, but we do not consider these channels in this
  work. See Harris and Sass (\protect\hyperlink{ref-harris}{2011}),
  Wayne et al. (\protect\hyperlink{ref-wayne}{2008}), and Boyd et al.
  (\protect\hyperlink{ref-boyd2009}{2009}).}.

In this paper, we consider pecuniary and non-pecuniary predictors of
various types of teacher churn in replicating the analyses of Hanushek,
Kain, and Rivkin (\protect\hyperlink{ref-hanushek}{2004}) (HKR) in a new
context (Wisconsin) and time horizon (2000 - 2010). The headline results
of Hanushek, Kain, and Rivkin (\protect\hyperlink{ref-hanushek}{2004})
were that ``teacher mobility is much more strongly related to
characteristics of the students, particularly race and achievement, than
to salary, although salary exerts a modest impact once compensating
differentials are taken into account.'' We confirm the pith of this
conclusion, namely that student characteristics are a much better
predictor of turnover than are wage differentials, though we come to
different conclusions regarding more specific points. To wit, while we
do find strong evidence that the socioeconomic makeup of a teacher's
district predicts turnover (and that there is heterogeneity in this
effect by teacher race), the evidence we find for the importance of
wages and student achievement is far from compelling.

We explore to the extent possible potential contributors to this
discrepancy in results; most salient are the measurable differences
between Texas, where Hanushek, Kain, and Rivkin
(\protect\hyperlink{ref-hanushek}{2004}) conduct their study, and
Wisconsin. Wisconsin is a largely rural state -- its largest
city/metropolitan area, Milwaukee, currently has roughly 600,000
residents (1,500,000 including the metropolitan area), making it around
the 30th-largest city in the United States\footnote{In fact, Milwaukee
  is the only city in Wisconsin considered to be ``large'' for NCES
  reporting purposes.}. By contrast, Texas has six cities larger than
this, with El Paso (\#6 in Texas) being the nearest in size to
Milwaukee. Though the non-urban parts of Texas are themselves sparsely
populated and distinctly rural, the more uniform lack of major
population centers in Wisconsin is likely to be reflected when
considerably different preferences among local vis-à-vis urban residents
for various aspects of potential teaching positions are aggregated.

To the end of exploring the pecuniary aspects of teacher turnover, we
start by expanding upon the efforts of Hanushek, Kain, and Rivkin
(\protect\hyperlink{ref-hanushek}{2004}) to infer teachers' tenure-wage
paths from teacher-level data on pay. Most unionized teachers are paid
according to a salary schedule (specifying wages as an increasing
function of tenure and certification) explicated in contracts
collectively bargained at the district level. With this easily-obtained
information in hand, teachers are able to infer their future potential
wage trajectories at their own and other potential district employers.
Lacking the physical contract faced by the teachers, an econometrician
armed only with administrative data reporting actual wages in a given
year must use some imputation techniques to deduce the underlying wage
structure. We explore the utility of natural Constrained B-Splines
(COBS) to this end. COBS are an enhanced version of the traditional
semiparametric splines technique enhanced by the ability to impose a
monotonicity constraint on the resultant curve which allows the fit to
incorporate more local information from nearby experience cells.

The fidelity and utility of the resulting fitted contract curves are
supreme. In both large and small districts, COBS produces a plausible
tenure-wage arc which enables us to examine counterfactual wage levels
for mobile teachers. By comparing the fit to a small number of wage
tables obtained from actual contracts, we also learn that using COBS may
be preferable to an attempt to use actual wage tables, as the
data-derived curves can reveal latent progress of teachers towards
further certification, an aspect which is commonly observed in salary
tables but rarely included in teacher-level data Thus,
empircally-derived curves can hew closer to the true wage path that
rationally forward-thinking teachers are likely to consider when mulling
over career options.

\TAG{END_INTRO}

\TAG{BEGIN_BODY}

\section{Literature Review}\label{literature-review}

Because the potential policy implications of turnover in the teaching
profession (from human capital and equity/distributional perspectives
both) are far-reaching and polypartisan, the literature on
turnover-related topics in education is extensive. As relates to this
paper, there are five broad (and often overlapping) categories of
inquiry: the relationship between turnover and wages, which has tended
to focus on ``opportunity wages'' outside of the field of education; the
relationship between turnover, school demographics, and other
nonpecuniary benefits, which has tended to focus on distributional
inequalities--whether teachers with certain characteristics are more or
less likely to be teaching certain disadvantaged groups; the
relationship between turnover and teacher quality as measured by student
performance, usually value added (VA); collective bargaining agreements
in education, focusing by and large on the implications (or lack
thereof) of seniority-preferential clauses; and the recent phenomenon of
specific retention incentives, the provisioning of wage bonuses to
teachers willing to teach in high-needs schools.

One of the earliest papers attempting to rigorously investigate turnover
was a panel study of teachers in Michigan by Murnane and Olsen
(\protect\hyperlink{ref-murnane}{1990}), who used college degree field
wages outside of education as opportunity wages, finding the expected
lower exit rate for teachers with higher wages in teaching relative to
the authors' defined alternative. Dolton and Van der Klaauw
(\protect\hyperlink{ref-dolton}{1999}) use panel data on university
graduates in the United Kingdom to estimate a competing risks model of
the decision to leave teaching entirely, finding results in line with
Murnane and Olsen (\protect\hyperlink{ref-murnane}{1990}). Returning to
panel studies in the US, Loeb and Page
(\protect\hyperlink{ref-loeb}{2000}) use PUMS data to get an idea of
teacher relative wages in many states and find that dropout rates fall
when teacher relative wages are high. Stinebrickner
(\protect\hyperlink{ref-stinebrickner}{2002}) also uses panel data (this
time NLS-72) to track both teachers and non-teachers, focusing in
particular on young teachers who leave the profession for long stints,
and finds that the best predictor of female exit is recent childbearing,
which is an important consideration for all work related to teacher
turnover because such a high percentage (76 nationwide) of teachers are
female. Lastly, Hanushek, Kain, and Rivkin
(\protect\hyperlink{ref-hanushek}{2004}) focuses on teachers in Texas
and emphasizes that the characteristics of students are much stronger
factors in predicting teacher exit than are wages (while also affirming
the statistical significance of pay).

While wages have been found consistently to have some measurable effect
on teacher turnover, it is impossible to explain within-district
migration (which constitutes a large portion of switching--as much as
50\%) through wage-only channels because contracts are fixed at the
district level. As such, another strand of literature has chosen to
focus on the nonpecuniary aspects of the decision to take a teaching
job--school environment/rapport, student enthusiasm, neighborhood
characteristics, etc.--usually by directing attention to a single
district so that any wage-based considerations are stifled, as is the
case for Boyd et al. (\protect\hyperlink{ref-boyd2005}{2005}) and Engel,
Jacob, and Curran (\protect\hyperlink{ref-engel}{2014}). Boyd et al.
(\protect\hyperlink{ref-boyd2005}{2005}) track early-career teachers in
New York City as they quit or transfer out of the city, and most
importantly finds that commuting time is an important, often overlooked
aspect of location preference. Engel, Jacob, and Curran
(\protect\hyperlink{ref-engel}{2014}) leverages a unique data set from
Chicago Public School job fairs which affords them a rather strong
measure of teachers' demand for vacancies, neutralizing the influence of
school administration's behavior on turnover (through poor match
selection or other means). The authors contribute evidence that the
school's neighborhood (perhaps due to ambient crime or other
reputational effects good and bad) is a better predictor of teachers'
preference than distance from home, going somewhat against the grain of
Boyd et al. (\protect\hyperlink{ref-boyd2005}{2005}). Scafidi, Sjoquist,
and Stinebrickner (\protect\hyperlink{ref-scafidi}{2007}) examine
statewide data from Georgia, but ignore wage effects, choosing instead
to focus on disentangling the contributions of low student achievement
and minority status to turnover; they find that minority status is the
more salient associate of teacher exit.

The key element missing from all of the above studies is perhaps the
most important consideration in the issue of teacher turnover--teacher
quality. None of the studies above have student-teacher matched data,
and so are unable to directly associate student outcomes with any given
teacher. If, with respect to any measure of quality you would like, we
find that transitioning teachers are identical to their replacements,
the issue of teacher turnover is not, in fact, much of an issue. Thus,
the recent trend in the literature to incorporate measures of teacher
quality (in large part made possible by a trend towards administrative
records allowing students to be linked to teachers and tracked over
time) in considerations of teacher turnover has made big strides in
addressing the most policy-relevant questions to be asked. The most
common and widely accepted measure of teacher quality is VA\footnote{The
  most commonly cited expositions on value-added, its validity, and so
  on are probably Rivkin, Hanushek, and Kain
  (\protect\hyperlink{ref-rivkin}{2005}), an extensive exploration of
  the predictive powers of empirical Bayes VA measures; and Chetty,
  Friedman, and Rockoff
  (\protect\hyperlink{ref-chettyI}{2014}\protect\hyperlink{ref-chettyI}{a})
  and Chetty, Friedman, and Rockoff
  (\protect\hyperlink{ref-chettyII}{2014}\protect\hyperlink{ref-chettyII}{b}),
  the largest-scale study of long-term inferences based on VA.} (in its
various guises), and the literature has begun to incorporate such
measures into studies of teacher turnover. Hanushek and Rivkin
(\protect\hyperlink{ref-hanushek2010}{2010}) consider VA as a measure of
teacher productivity, and ask if common results of labor search theory
(namely that turnover falls with tenure and that turnover is negatively
associated with match-specific productivity) continue to hold in the
education labor market. In fact, the authors find that the teachers most
likely to switch schools are those with low measured match quality, and
especially that those who leave teaching entirely are those with the
lowest match quality. The results are more pronounced for schools with
high proportions of low-SES students, which has strong policy
implications, as it appears the best teachers in high needs schools are
the least likely to change jobs. Goldhaber, Gross, and Player
(\protect\hyperlink{ref-goldhaber2007}{2007}) performs a similar
analysis with the longitudinal data of North Carolina and comes to
similar conclusions, strengthening the robustness of the results.
Lastly, Goldhaber, Lavery, and Theobald
(\protect\hyperlink{ref-goldhaber2015}{2015}) examine the inequity in
the distribution of teacher quality by high-needs groups in Washington
state, and find that for all three measures of quality (teacher
experience, licensure exam score, and VA), the distribution of teachers
favors the less needy (as measured by free/reduced-price lunch status,
minority status, and low prior academic achievement).

The aforementioned papers have tended to keep the collective bargaining
aspect of salary determination for teachers out of the spotlight, if
largely for reasons of data restrictions. Nevertheless, it stands to
reason to believe that the rigid structure of union-negotiated contracts
could serve to contribute in a large way to teacher turnover. Ballou and
Podgursky (\protect\hyperlink{ref-ballou}{2002}) give much descriptive
evidence of the shape of the wage-tenure profile, rooted in a data set
collected by the Department of Defense and published by the AFT. They
find that seniority premia in education largely mirror those in more
traditional white collar professins, that steeper profiles are
associated with less turnover, and that district financial and
demographic conditions alone are insufficient to explain variation in
contracts. Another common (and recently quite controversial, as
evidenced by the contention in the ongoing contract negotiations in
Philadelphia) feature of union-negotiated teacher contracts are
seniority priviliges--preferential treatments granted to teachers in
voluntary and involuntary transfers. Moe
(\protect\hyperlink{ref-moe}{2006}) codes contracts from 158 districts
in California according to the strength of seniority rights therein
guaranteed to teachers and finds that such rights are associated with
the distribution of teachers across schools (measuring quality as
experience and certification) in a way that serves to harm minorities.
Revisiting California with a slightly different sample and definition of
the ``determinacy'' of the contracts with respect to seniority, Koski
and Horng (\protect\hyperlink{ref-koski}{2007}) come to the opposite
conclusion--that there is no such relationship. As a rebuttal, Anzia and
Moe (\protect\hyperlink{ref-anzia}{2014}) pin the difference in results
on the exclusion in Moe (\protect\hyperlink{ref-moe}{2006}) of small
school districts, where it appears that the entrenchment of bureaucracy
falters and the rigidity of contract language wane, a claim which they
support by repeating their analysis with the inclusion of an interaction
for district size--indeed, for small districts the result of Koski and
Horng (\protect\hyperlink{ref-koski}{2007}) holds, while the insight of
Moe (\protect\hyperlink{ref-moe}{2006}) holds in larger districts.
Cohen-Vogel, Feng, and Osborne-Lampkin
(\protect\hyperlink{ref-cohenvogel}{2013}) use data from Florida and
their results align with those of Koski and Horng
(\protect\hyperlink{ref-koski}{2007}) (though they neglect to nuance
their results by district size).

Finally, an emerging strand of literature is looking at the potential
for transfer bonuses and retention incentives to positively affect
student outcomes. Fulbeck (\protect\hyperlink{ref-fulbeck}{2014})
analyzes a scheme in place in Denver whereby teachers who choose to
transfer to high-needs schools (low-performing) are given recurring
bonus pay, and those initially stationed there are given retention
incentives. She concludes that recipients of incentives are
significantly less likely to switch jobs, as driven by a reduction in
district exit rates and especially by teachers whos incentive payments
exceed \$5,000. Glazerman et al.
(\protect\hyperlink{ref-glazerman}{2013}) evaluate the Talent Transfer
Initiative, a randomized controlled trial conducted in 10 districts
whereby high-performance teachers were given \$20,000 over the course of
two years as reward for transferring the identified high-needs schools,
and conclude that there were significant effects on teacher retention as
well as on student outcomes.

Two highly germane papers investigate the impact in Wisconsin on
teachers of Governor Scott Walker's flagship policy, Act 10, which
severely limited the scope for collective bargaining in the state.
Litten (\protect\hyperlink{ref-litten}{2016}) uses differences in
contract renewal dates surrounding the policy's enactment to evince the
effect of unionization on teachers' wages, and finds the lack of union
bargaining power reduced teacher compensation by 8\%. Biasi
(\protect\hyperlink{ref-biasi}{2017}) constructs value-added measures
from grade-level test results and concludes that the move to
individually-negotiated salaries in some districts had a significant
impact on teacher quality and student outcomes in such districts, while
also cautioning that most of these gains are competition-based, so that
scaling up the system state-wide would have an impact limited to a boost
from the exit of low-quality teachers.

\section{Data}\label{data}

The State of Wisconsin's Department of Public Instruction (DPI) releases
annual Salary, Position \& Demographic reports through the WISEstaff
data collection system. These reports represent ``a point-in-time
collection of all staff members in public schools as of the 3rd Friday
of September\ldots{}'' (Public Instruction
\protect\hyperlink{ref-dpi}{2017}\protect\hyperlink{ref-dpi}{b}), and
will serve as the primary source of data on teachers in this paper. Data
are available at the position-teacher level cross-sectionally, with each
entry in a given year corresponding to one of possibly several
positions/assignments held by each school district employee\footnote{Many
  teachers (and other district employees) serve in multiple roles within
  a school/district, for example as a coach, part-time program aide, or
  department head. Each of these is filed as a separate observation in
  the DPI system, though salary information is given at the teacher as
  opposed to the assignment level.}. Identifiers in each file permit
unique identification of an employee within a given year, but this
identifier does not follow teachers between years\footnote{From
  AY2011-12, a field called the File Number appears to allow
  longitudinal tracking of teachers. We use this in part to validate the
  matching algorithm; see the Appendix.}. To overcome this substantial
hurdle to identifying teacher mobility, data are first fed through the
matching algorithm described in further detail in the Appendix.
Essentially, we are aided by the availability of various imperfect
identifiers which should be more stable over time, most crucially
teachers' first and last names and year of birth. By building on these
covariates and incorporating some limited fuzzy matching techniques, we
construct a panel of teachers spanning the 1994-95 academic year (AY)
through AY2015-16\footnote{For brevity, we herein refer to academic
  years by the spring year, e.g., AY2003-04 will be simply 2004.}
consisting of 3,588,614 teacher-position-year observations. The matching
algorithm necessitates elimination of 26,304 (0.7\%) observations over
all 21 years on account of belonging to teachers who could not be
uniquely identified in a given year of data due to exact overlap of
their first name, last name, and birth year fields with another teacher
in the data\footnote{Technically, we use a slightly modified version of
  the name strings in making these eliminations which, for example,
  eliminates initials -- see Appendix.}.

Specific to the exercise at hand, with data reliability and precision in
mind, we make the following series of further restrictions on the data.
The introduction of Wisconsin Act 10 introduced a substantial structural
break in the labor market for Wisconsin teachers, so we include only
data from 2000-2010 to avoid conflating the effects of this policy on
teacher turnover with the earlier functioning of the labor market (i.e.,
we do not want to mix the results from distinct equilibria of the
teacher labor market, but would instead prefer to analyze the pre- and
post-Act-10 markets separately). We drop all employees who are not
full-time, full-year regular teachers of a major core subject
(all-purpose elementary teachers or English/Math) at a single regular
public school with a Bachelor's or Master's degree and fewer than 35
years' recorded experience; taken together, these restrictions eliminate
79\% of employees, the lion's share of which come from eliminating
substitutes/support staff and teachers of on-core subjects\footnote{\label{ftn:pos_code}We
  also eliminate any teacher who appears in any role besides ``Teacher''
  in any year. In particular, this eliminates a nontrivial number of
  educators who begin their career with an ``ease-in'' period, take a
  mid-career ``leave'' speckled with a transition to substitute teaching
  -- perhaps during their child's infancy -- or end it with a ``soft
  retirement'' period, during which they act as a substitute teacher at
  some point in the midst of a career otherwise focused on teaching.
  Such teachers often have part-time roles at several local schools,
  which introduces sufficient ambiguity in the definition of mobility so
  as to obscure interpretation of results, so we opt for a stricter
  definition of full-time teaching than is completely necessary.}. We
then eliminate teachers with missing information on their subsequent
school or district and teachers with instability in their recorded
ethnicity, as well as teachers not categorized as white, black, or
Hispanic, eliminating a further 0.2\% of all employees\footnote{Wisconsin
  teachers are predominantly white (96\%). As noted in the Appendix, we
  also use the panel data to correct noise found in recorded ethnicity
  and gender over time for some teachers. In the final sample, 427 and
  345 teachers had their ethnicity and gender (respectively) adjusted in
  some year(s).}. Finally, we drop teachers' multiple positions by
keeping only the highest-intensity position for each teacher, as
measured by full-time equivalency, resulting in a final count of 282,797
teacher-year observations -- 49,325s in 449 districts and 2,296.

The data used for the incorporation of counterfactual salary
calculations is largely the same, but with a few noteworthy differences.
First, as noted in Footnote \ref{ftn:pos_code}, the main turnover data
eliminated some teachers who transitioned in and out of being
categorized as a full-time teacher due to the muddling effects thereof
on defining turnover. This concern not being relevant to constructing
the salary schedules, it is not imposed for this data. Next, because all
regular teachers are covered by the same collective bargaining
agreement, the salary imputation data is less restrictive with respect
to the subject codes excluded from the data, and generally includes any
teacher not in special education. This data also ignores instability in
recorded gender and ethnicity within a teacher.

Finally, the salary data loses observations that are present in the
turnover data based on a series of cuts which are either required for
COBS to function, or else substantially increase the reliability of its
output. The most noteworthy/far-reaching of these numerical restrictions
is to eliminate any teachers working in districts where there are not at
least 20 total teachers in each degree track for that year. While
ultimately arbitrary, this number is reasonable to limit the potential
effect of an individual teacher on an exercise determining 35 levels of
pay with minimal functional form restrictions. The other numerical flags
require both the BA \& MA track to be represented at a district, for at
least 7 distinct levels of experience to be represented within a degree
track, and for at least 5 unique values of the two measures of pay
(salary and fringe benefits) to be available in each degree track; all
teachers at districts failing at least one of these tests is dropped.

The sum total of all of these restrictions leaves us with an analysis
sample of 356,265 teacher-year observations to be used to estimate pay
scales, made up of 65,069 individual teachers in 209 districts over 11
years. In total, there is sufficient data to fit 3,708
\(y_t(\tau, c, d)\) curves, an average of roughly 100 observations per
curve. Ultimately around 22\% of teachers have missing salary
information\footnote{HKR include like-minded restrictions, but combine
  teachers of different certification within an experience level,
  despite the headline importance of this factor to teacher pay --
  median pay at a given level of experience is on average 17 higher for
  those with a Master's degree.}, mostly in rural districts or other
districts with only one or two schools and a small number of students.

We supplement the WISEstaff data set in several ways to incorporate
information about other characteristics of schools and districts in
Wisconsin. To get school- and district-level measures of socioeconomic
makeup (percentage of students who are black or Hispanic or eligible for
free/reduced lunches) and community type/urbanicity, we tap the Universe
Surveys from the National Center for Education Statistics' Common Core
of Data, which provide this information on a yearly basis for all years
in the study\footnote{The method of recording urbanicity by the Common
  Core switched from being ``metropolitan-centric'' to being
  ``urban-centric'' for Wisconsin from 2006 (Sable
  \protect\hyperlink{ref-sable}{2009}). We map the corresponding codes
  to match those used by HKR as well as possible, and use the data file
  from 2006, which has both types of code for all US districts, to
  confirm that the pre- and post-2006 correspondence is by-and-large
  working as intended. For a small number of districts/schools with
  missing urbanicity codes in certain years, we use information about
  that entity from other years to inform urbanicity.}\textsuperscript{,}\footnote{Further,
  the WKCE data does not include a standard deviation field even in
  those years when the school average scale score is available,
  precluding any attempt to standardize test scores and put the data
  here on equal footing with that of HKR.}. At the district level, we
also use this data to compute class size and the size of the student
body.

Lastly, we turn to DPI's public data again to get school- and
district-level performance metrics. While Hanushek, Kain, and Rivkin
(\protect\hyperlink{ref-hanushek}{2004}) were able to obtain school- and
district-level average scale scores on a standardized test in Texas,
such a metric is not publicly available in Wisconsin for all years.
Instead, we calculate student proficiency rates for each school and
district as the percentage of test-takers deemed to be at grade level in
mathematics or reading in a given year on the Wisconsin Knowledge and
Concepts Examination (WKCE), which is administered to 4th, 8th, and
10th-grade students.

\section{Salary Scale Imputation with Constrained
B-Splines}\label{salary-scale-imputation-with-constrained-b-splines}

For many years, the ubiquitous characteristic of collectively bargained
teachers' contracts has been the salary table, which gives a mapping
from the calendar year, a teacher's experience (their length of tenure
at the current district), and their certification (typically Master's
vs.~Bachelor's degree) to their wage. This table gives current teachers
a clear understanding of how their pay will advance as a function of
their labor inputs, and thereby gives forward-looking potential teachers
and potential migrant teachers a clear understanding of their would-be
pay arcs under a district-switching decision-making framework,
especially given that this information is typically openly available.

It would behoove an econometrician seeking to understand education labor
market dynamics, then, to incorporate this information on future pay
into their statistical modeling framework. Unfortunately, this data is
typically not available in a format lending itself to easy analysis at
scale -- whether locked inside idiosyncratically formatted and
sporadically-available contract PDFs or hidden behind large-scale
freedom of information act inquiries, the temporal and financial costs
of scraping such data into a usable form can be substantial.

Much more common in empirical settings is access to teacher-year-level
salary data of the form
\(y_{i, t} = y(\tau_{i, t}, c_{i, t}, d_{i, t}) + \varepsilon_{i, t}\),
where \(\tau_{i,t}\), \(c_{i, t}\) and \(d_{i, t}\) are the tenure,
certification, and district of teacher \(i\) in year \(t\), and
\(\varepsilon_{i, t}\) represents unaccounted factors affecting the wage
(e.g., not all teachers work full time, some teachers split their time
among duties yielding different pay levels, and many teachers supplement
their income with additional duties like coaching). Here we consider one
approach and some empirical lessons for trying to estimate the
underlying mapping \(y(\tau, c, d)\) from such data.

There are a multitude of inference/imputation techniques suitable to the
inference of a latent function of unknown parametric form available in
the statistician/econometrician's palette. The powerful flexibility of
nonparametric approaches (local regression, splines, Random Fourier
Feature expansions) is a double-edged sword; as it happens, in this
particular setting, even if we know linearity is not a reasonable
functional form restriction, we do know some very basic properties of
the underlying tenure-wage curves that will be violated in general by
uninformed estimation techniques. In particular, we know that such
tenure-wage curves are non-decreasing and that they are non-negative,
i.e., \(y(\tau', c, d) >= y(\tau, c, d)\) whenever \(\tau' >= \tau\),
and \(y(0, c, d) \geq 0\).

He and Ng (\protect\hyperlink{ref-he}{1999}) introduce a linear
programming approach to incorporating monotonicity, curvature, and
pointwise restraints to quantile regression spline estimation
techniques, and Ng and Maechler (\protect\hyperlink{ref-ng}{2007})
present an overview of the R package \texttt{cobs} which gives an
efficient implementation of this approach (COBS standing for Constrained
B-Splines; B-splines are computationally-efficient basis functions for
degree \(k\) splines). The basic idea of quantile regression spline
estimation is to swap out the standard squared loss function for a
quantile-dependent weighted absolute loss function to target conditional
quantiles instead of conditional means. Monotonicity, point, and
curvature restrictions enter as penalized terms to the objective
function; \texttt{cobs} expresses this in a fashion which facilitates
the application of standard linear programming techniques for
efficiency, and handles internally the issues of knot selection and
penalty parameter assignment through cross-validation.

We implement and fine-tune this general approach with an eye to being as
minimally-invasive as possible. The first innovation is required by the
poor performance of standard COBS fit in extrapolation. Data sparsity in
smaller districts means that it is often the case that only a small
range of \(\tau\) values are observed in a given
year-certification-district. Monotonicty constraints are only built into
the B-spline routine internally; the underlying basis functions may
produce decreasing fits outside the observed range of data. To overcome
this, we take a cue from the literature tackling Runge's Phenomenon
(Runge \protect\hyperlink{ref-runge}{1901}), wherein polynomial
approximations tend to exhibit extreme oscillations in extrapolation.
This issue is one of the motivations behind natural cubic/smoothing
splines (see, e.g., Friedman, Hastie, and Tibshirani
\protect\hyperlink{ref-friedman}{2001}; Wahba
\protect\hyperlink{ref-wahba}{1990}; Green and Silverman
\protect\hyperlink{ref-green}{1993}; or de Boor
\protect\hyperlink{ref-deboor}{1978}), which handle this issue by using
a simple linear basis function outside the outermost interpolating
knots. We incorporate this technique of linear extension only when
necessary by testing the COBS fit for monotonicity; \(\tau\) values
failing this constraint are replaced by extending the final
non-decreasing fit values through the end of the range of extrapolation.

Next, a major shortcoming of COBS for this context is its limit to
one-dimensional spline fits; while techniques for nonparametric B-spline
fits are available in arbitrary dimensions (see de Boor
\protect\hyperlink{ref-deboor}{1978}), at present COBS is only capable
of imposing monotonicity on one dimension of a curve. In our context,
however, \(y(\tau, c)\) is increasing not only with respect to \(\tau\),
but also with respect to \(c\) (as, without fail until only very
recently in Wisconsin, certification was rewarded with a Master's
premium, typically a percentage increase in wage). One solution would be
to generalize the implementation of COBS to handle a second dimension by
simply adding penalty terms along this dimesion\footnote{Not to mention
  the empirical reality that the two-lane dichotomy is in fact false --
  it is very common, nearly ubiquitous, for contracts to offer separate
  lanes for teachers with the same completed certification, but
  different levels of progress towards completing further certification.
  As this dimension is impossible to glean from our data, we exclude it
  from the imputation exercise.}. We abandon this approach because of
the categorical nature of the certification dimension -- there are not
numerical units to the difference between having a Master's
vs.~Bachelor's degree. The assignment of such a number required by this
approach would itself become an implementation hyperparameter, meaning
that the ultimate fit would itself be sensitive to the particular choice
of continuous representation.

Instead, we use a two-step procedure to fit the Bachelor's and Master's
pay tracks in serial. In the first step, we fit the Bachelor's career
track as a typical one-dimensional COBS fit. In the second step, we
first construct Master's premia for each observation by subtracting out
the predicted Bachelor's pay corresponding to each observed level of
tenure for a teacher with a Master's degree. We then use COBS to fit a
non-decreasing Master's premium curve over all tenure levels on these
residuals, before finally adding the Master's premium and Bachelor's fit
curves to get the overall Master's fit curve. Monotonicity of the result
is guaranteed by forcing upwards monotonicity on the Master's premium, a
restriction in line with the empirical observation that Master's degree
pay is often simply a fixed-percentage rise over the corresponding
Bachelor's pay.

Our implementation was also aided by the imposition of weak concavity on
the \(y(\tau, BA)\). While not a theoretically-assured functional form
restriction\footnote{In fact, in reality tenure-wage curves are often
  piecewise convex -- year-over-year rises are specified as a percentage
  bump which eventually levels off to either linear increase or maxes
  out and flattens. Nevertheless, the degree of convexity in that
  section of the curve tends to be low, which leads COBS to fit a good
  linear approximation there. The lack of a concavity restriction on the
  Master's pay track allows fit curves which fit this pattern for this
  lane.}, concavity improves the goodness of fit notably. Small-sample
district-level observations and simple reduced-form regressions of wages
versus quadratic forms in experience support this shape's validity. A
variety of contracts obtained from a database for teachers in nearby
Michigan also meet this condition, and the decrease in marginal returns
to experience is also commonly found in the wider study of labor
markets\footnote{See, e.g., Heckman, Lochner, and Todd
  (\protect\hyperlink{ref-heckman}{2003}).}. We do not impose this
restriction on the fit for the Master's premium (the only restrictions
there being non-negativity at 0 and, as mentioned, an increasing
relationship with tenure).

\begin{figure}[htbp]
\centering
\includegraphics{figures/milwaukee-1.pdf}
\caption{\label{fig:mwk}Pay in Milwaukee, 2003-2006}
\end{figure}

As an illustrative example of the patterns in the data we wish to
quantify and formalize, we turn briefly now to Milwaukee Public Schools,
the largest district in Wisconsin with roughly 32,387.73 teachers per
year. Figure \ref{fig:mwk} depicts key moments of the empirical
distribution of salary in 4 years at Milwaukee Public Schools, broken
down by tenure and certification. The central lines on each plot
(Bachelor's pay track in blue, Master's pay track in red) are the
empirical median levels of pay, and thus give a rough approximation to
\(y(\tau, c)\). The dashed-line intervals on either side represent the
25th and 75th percentiles.

Notably, these intervals and the medians themselves tend to get quite
noisy at later stages in the career, especially for the Bachelor's
track. This fact that reflects the almost universal certification of
teachers by about 15 years into their career. This is reflected in the
bar graph below each set of curves, which shows the distribution of
teachers in each certification track by tenure. Almost all new teachers
start with only a Bachelor's degree; the relative presence of Master's
degrees grows over time as more teachers certify mid-career.

We can also note two more key empirical facts from this plot. First, the
vanishing presence of teachers in both certification tracks leads the
empirical median to be a poor approximation of \(y(\tau, c)\) since it
frequently fails to respect the fundamental monotonicity constraint
discussed above. With respect to tenure, this tends to affect the
Bachelor's track later in the career as more teachers certify, and the
Master's track very early in the career before teachers certify. The
monotonicity with respect to \(c\) of the median wage is mostly
maintained here for Milwaukee, but this is not always the case; our
estimation procedure is thus careful to impose these restrictions
internally.

Second, structural breaks are an important empirical phenomenon in this
context. Each time a contract is renegotiated at a district, the
tenure-wage curves can potentially change shape dramatically. It is with
this in mind that we refrain, given our ignorance with respect to when
such structural breaks occur, from combining information from adjacent
years in fitting a given year's curve, an approach which would
substantially enhance the statistical power available to fit contracts
for sparsely-populated districts. Such a structural break is apparent in
Milwaukee, for example, between 2003 and 2004 and between 2005 and 2006,
where the shape of the Master's pay scale has shifted notably. While we
eschew, for example, full Bayesian estimation of structural breaks in a
given district, such techniques are applicable and worthy of future
exploration.

\subsection{Goodness of Fit}\label{goodness-of-fit}

Returning to the motivating example illustrated in Figure \ref{fig:mwk},
we turn first to the performance in Milwaukee, where, given the
relatively large sample size, performance is expected to be very good.
Indeed this is the case, as seen in Figure \ref{fig:mwk_fit}. The COBS
fit has retained all the salient features of the empirical median return
to experience and certification, while simultaneously improving over
this nonparametric conditional median by ironing out nonmonotonicities
found empirically as a result of small-sample bias.

\begin{figure}[htbp]
\centering
\includegraphics{figures/milwaukee_fit-1.pdf}
\caption{\label{fig:mwk_fit}Estimation Results for Milwaukee, 2003-2006}
\end{figure}

Perhaps more telling is the goodness of fit in minimally small
districts. Four such examples are featured in Figure
\ref{fig:small_fit}. These four districts just barely satisfy the sample
restriction that at least 20 teachers be present in both the BA and MA
pay track (each has fewer than 42 teachers, and only in a single year);
for this reason, rather than plot the empirical median, we simply
present the full distribution of wage, experience, and certification in
these districts. Here again the COBS fit captures the essence of the
wage-tenure curve even in these sparsely-staffed districts. Both
Montello and Manawa evince the importance of the non-negativity
constraint on extrapolated values of the Master's premium -- given the
absence of teachers so certified prior to the fifth year of experience,
some supplementary discipline is necessary to prevent the tail of this
curve from dipping below that for the Bachelor's lane.

\begin{figure}[htbp]
\centering
\includegraphics{figures/small_dist_fit-1.pdf}
\caption{\label{fig:small_fit}Estimation Results for Selected Sparse
Districts}
\end{figure}

A final check on the validity of the imputation procedure would be to
compare fit schedules side-by-side with the true schedules, e.g.~through
root mean-squared error. As mentioned, this is typically a difficult
undertaking on a mass scale since the true schedules may be hard to come
by in a parseable electronic format. Usually, a smaller-scale version of
this exercise would be possible through sampling, say, 5-10\% of
districts at random and spending the time to extract actual schedules by
hand for this purpose. Unfortunately, with the passage of Act 10 and the
abandonment of collective bargaining in many districts, electronic
copies of legacy contracts became hard to come by -- none of the large
districts we contacted (nor their former union representatives) had
access to old copies of contracts they were willing to share, nor could
we find any but a very small number of these contracts online. We
present here the comparison of COBS-produced fit to true schedule in
three district-year combinations for which we could actually obtain the
true schedule\footnote{These contracts and a few others from outside of
  the study time frame are available upon request.}.

\begin{figure}[htbp]
\centering
\includegraphics{figures/true_schedule_fit-1.pdf}
\caption{\label{fig:true_fit}Comparison of True Contracted Schedule with
Output of Imputation}
\end{figure}

As seen in Figure \ref{fig:true_fit}, the resulting fit is generally
superb\footnote{In terms of objective measures of the fit, the mean
  absolute error is \$1,762, while the overall median error is \$812.
  This is evidence against the assumption built into the COBS routine of
  0-median errors \(\varepsilon_{i, t}\), and is understandable -- it is
  not uncommon for teachers to earn supplementary pay from coaching or
  extra teaching duties that would push them above their
  salary-schedule-dictated pay grade. With a more complete set of
  training data, one could potentially account for this by treating the
  quantile of the data targeted by COBS (.5 by default, i.e., COBS is
  median-targeted) as a hyperparameter to be fit by cross-validation to
  prevent overfitting (see Stone \protect\hyperlink{ref-stone}{1974} or
  Friedman, Hastie, and Tibshirani
  (\protect\hyperlink{ref-friedman}{2001})).}. Only for the Bachelor's
track in Monona Grove in 2009 does the COBS-fit curve depart
substantially from the true contracted schedule. Moreover, this
departure is likely attributable to the oversimplification taken in this
paper of restricting pay to follow only two ``lanes'' (Bachelor's and
Master's degrees), when in reality districts often differentiate among
holders of these degrees by rewarding those with more credit-hours of
supplementary coursework under their belts in pursuit of continued
learning or a higher degree -- in fact, such coursework is often
required of Bachelor's-certified teachers, which means it is likely that
the later-career Bachelor-certified teachers observed in Monona Grove
are actually being paid according to a higher lane. This is exactly what
is depicted by the gray lines on the Monona Grove plot, which show the
BA+12 and BA+24 lanes are more representative of instructors at the
later stages of their career in Monona Grove (the data lack any way of
detecting a given teacher's extracurricular credit accumulation).

We take the above as strongly affirming the utility of COBS as a tool
for constructing wage-tenure curves from teacher-level salary data. It
is able to gloss over noise-induced non-monotonicities in the empirical
median, not just with the rich data found in urban districts, but also
in sparsely-populated districts. Moreover, as explored in the case of
Monona Grove School District, COBS can be seen as doing a good job of
capturing an aspect of the data which is still latent (namely, the
degree of progress towards further certification), and of being closer
to the ``true'' schedules that teachers use to make mobility decisions
(since it is likely that teachers are able to anticipate extra income
from holding multiple roles and factor this into their assessment of a
wage offer). Lastly, the COBS routine is computationally attractive --
embarassingly parallelizable and implemented very efficiently, the whole
routine runs in a few minutes.

\section{Turnover in Wisconsin}\label{turnover-in-wisconsin}

\begin{table}[htbp]
\centering
\begin{tabular}{p{.12\linewidth}p{.14\linewidth}p{.17\linewidth}p{.10\linewidth}p{.17\linewidth}R{.13}}
  \hline
 & \multicolumn{4}{c}{Percent of Teachers Who} & \\ \cline{2-5}
Teacher Experience & Remain in Same School & Change Schools Within District & Switch Districts & Exit Wisconsin \mbox{Public Schools} & Number of Teachers \\ 
  \hline
1-3 years & 79.7 & 7.1 & 6.0 & 7.2 & 41,042 \\ 
  4-6 years & 86.6 & 5.6 & 3.3 & 4.5 & 37,770 \\ 
  7-11 years & 90.6 & 5.0 & 1.8 & 2.6 & 54,623 \\ 
  12-30 years & 92.4 & 4.1 & 0.6 & 2.9 & 129,002 \\ 
  >30 years & 80.3 & 6.4 & 1.1 & 12.2 & 20,360 \\ 
  All & 88.6 & 5.1 & 2.0 & 4.3 & 282,797 \\ 
   \hline
\end{tabular}
\caption{Year-to-year Transitions of Teachers by Experience, 2000-10} 
\label{tbl:move_by_exp}
\end{table}

We move now to the core focus of our analysis, examining the
distinguishing features of turnover in the teacher labor market in
Wisconsin. Table \ref{tbl:move_by_exp} replicates Table 1 of Hanushek,
Kain, and Rivkin (\protect\hyperlink{ref-hanushek}{2004}), and as HKR
found in Texas, most turnover in Wisconsin is happening within districts
and out of the profession\footnote{This and subsequent analyses were
  greatly facilitated by several facilities of the R programming
  language, for which due credit must be given to R Core Team
  (\protect\hyperlink{ref-r}{2016}), RStudio Team
  (\protect\hyperlink{ref-rstudio}{2017}), Dowle and Srinivasan
  (\protect\hyperlink{ref-dowle}{2017}), Xie
  (\protect\hyperlink{ref-xie}{2016}), Leifeld
  (\protect\hyperlink{ref-leifeld}{2013}), Dahl
  (\protect\hyperlink{ref-dahl}{2009}), Henningsen and Toomet
  (\protect\hyperlink{ref-henningsen}{2011}), Zeileis and Hothorn
  (\protect\hyperlink{ref-zeileis2002}{2002}), Zeileis
  (\protect\hyperlink{ref-zeileis2004}{2004}), Zeileis
  (\protect\hyperlink{ref-zeileis2006}{2006}) and Croissant
  (\protect\hyperlink{ref-croissant}{2012}).}\textsuperscript{,}\footnote{We
  also note that some ``turnover'' identified by teachers not appearing
  at the same school in the following year is in fact spurious -- Public
  Instruction (\protect\hyperlink{ref-dpi_name_change}{2011}) identifies
  a number of instances of school districts merging during the timeframe
  of our analysis and hence disappearing from the data altogether. We
  take care to reset the district and school switch identifiers off for
  these 82 teachers if they appear in the newly-formed district in the
  subsequent period.}. In Wisconsin, the fraction of teachers
transitioning among districts is vanishingly small after a ``burn-in''
period of roughly 6 years -- only 1\% of such teachers do so (compared
with 3.1\% for the comparable group in HKR), but is still relatively
highest among the youngest teachers -- roughly twice as high for the
``probationary'' teachers (1-3 years' experience) as for teachers with
7-11 years' experience in both states.

By contrast, movement patterns within districts in the two states are
very similar, lending weight to teachers ``earning their stripes''
within a district to be able to choose the best schools as a privilege
of seniority. As expected, we also observe a U-shaped pattern in
teachers exiting Wisconsin public schools, which jibes with there being
two types of quits. Early-career quitters change to private schools,
change state of residence, or change professions; late-career quitters
retire -- especially evident among teachers with more than 30 years'
experience, a group which sees a mass exodus of fully 10 percent of its
teachers annually. Results not included here break down the exit rates
by experience level, where this dichotomy is even more dramatic --
first-year exit rates are about 8 percent and quickly level off at
around 2 percent before spiking again past around 25 years.

As examined further below, the low rate of switches between districts
appears to be owing to the generally more rural nature of Wisconsin
vis-à-vis. Texas. To wit, Milwaukee is the only major urban area in the
state, and its population (2010 Census) of 594,833 would rank 7th in
Texas. This means that two major types of movers in the HKR data --
Large Urban - Large Urban and Suburban - Large Urban -- are limited
within the state to ending up in a relatively minor metropolitan area.
HKR don't provide any results disaggregated by city, precluding any
attempts to compare these numbers more comparably to those that would
obtain from eliminating the largest cities in Texas.

\begin{sidewaystable}[htbp]
\centering
\begin{tabular}{lrrrrrrr}
  \hline
 & \multicolumn{4}{c}{\multirow{2}{*}{Percent of Teachers Who Move to}} & \multirow{4}{*}{\parbox{0.09\linewidth}{Number Teachers Changing Districts}} & \multirow{4}{*}{\parbox{0.07\linewidth}{Percent of Origin Teachers}} & \multirow{4}{*}{\parbox{0.09\linewidth}{Change in Share of Teachers 2000-06}}\\
 & \multicolumn{4}{c}{} & & & \\ \cline{2-5}
& & & & & & & \\
Origin Community & Large Urban & Small Urban & Suburban & Rural &  &  &  \\ 
  \hline
I. All teachers & & & & & & & \\
\quad Large Urban & 17.4 & 15.8 & 48.7 & 18.1 & 819 & 2.7 & 0.4\% \\ 
  \quad Small Urban & 3.7 & 13.4 & 44.8 & 38.1 & 640 & 1.2 & 0.1\% \\ 
  \quad Suburban & 3.4 & 16.0 & 44.2 & 36.4 & 1,408 & 1.9 & 3.6\% \\ 
  \quad Rural & 0.6 & 11.2 & 24.2 & 63.9 & 2,794 & 2.2 & -4.1\% \\ 
\multicolumn{3}{l}{II. Probationary teachers (1-3 years experience)} & & & & & \\
  \quad Large Urban & 15.8 & 17.9 & 47.6 & 18.8 & 437 & 5.0 &  \\ 
  \quad Small Urban & 5.1 & 14.0 & 46.2 & 34.8 & 271 & 3.9 &  \\ 
  \quad Suburban & 4.3 & 16.2 & 41.0 & 38.4 & 561 & 5.5 &  \\ 
  \quad Rural & 0.3 & 10.9 & 24.4 & 64.5 & 1,204 & 8.0 &  \\ 
   \hline
\end{tabular}
\caption{Destination Community Type for Teachers Changing Districts, by Origin Community Type and Teacher Experience Level} 
\label{tbl:markov}
\end{sidewaystable}

Moving from the aggregate numbers to begin to examine heterogeneity in
turnover, Table \ref{tbl:markov} replicates HKR Table 2, and
reverberates its most important conclusions. HKR argue that there is
little support for the idea that scores of young teachers are using
large urban schools as a training ground before ``settling down'' with
easier assignments in the suburb, based on the general low level of
turnover from Large Urban districts. We affirm the scarcity of
transitions from districts in Milwaukee, while also noting that such a
path is certainly present, as evidenced by the plurality of those who do
leave Large Urban districts ending up in a Surburban district in both
settings. HKR also observe that the likelihoods of remaining in the same
school and of quitting are roughly the same for urban and suburban
teachers, an observation which we can confirm in Wisconsin. We further
note that while Table \ref{tbl:markov} only presents a cross-sectional
picture, the career-long trend reaffirms this -- only 3.2\% of teachers
starting their careers at a large urban district ever work at a suburban
district. Lastly, we echo the suggestion of HKR that this phenomenon
cannot be driven purely by demand-side constraints -- in our time period
of observation, we observe only 1,459 urban teachers change districts,
whereas 3,211 teachers were hired in suburban districts, though of
course this does not rule out arguments based for example on stricter
screening of applicants transferring from urban districts.

We note, however, that though tales of flight from troubled urban
districts are apparently anecdotal, they are far from apocryphal. To
wit, while 50 percent of districts have a net inflow (arrivals less
departures) of four or fewer teachers (in absolute value), Milwaukee's
net outflow was 533 teachers, and the five highest-inflow districts, all
suburbs of Milwaukee or districts adjacent the main university town of
Madison, saw in total an inflow of 229 teachers in this time. This being
a two-sided market, this state of affairs is perhaps largely
attributable to the dynamic nature of student populations at these
districts -- but these, as well, are reflective of the appeal of the
districts to parents (and teachers as parents).

\begin{figure}[htbp]
\centering
\includegraphics{figures/tx_wi_urb_dist-1.pdf}
\caption{\label{fig:ti_wi_urb}Comparison of the Prevalence of Different
Community Types}
\end{figure}

As mentioned in the discussion of Table \ref{tbl:move_by_exp}, the major
difference with respect to quantities observed in Texas appears to be
driven by differences in the urban landscape between Texas and
Wisconsin\footnote{We also note a difference (as found in Table
  \ref{tbl:markov}) in the relative shift in population among community
  types between the two states -- Texas observed dramatic changes in its
  community type distribution over the period of study of only 4 years,
  while Wisconsin only saw some movement from Rural to Suburban
  communities over a longer period of 11 years.}. This is supported by
the overall similarity of magnitudes of transition rates to community
types besides Large Urban in the two papers. Figure \ref{fig:ti_wi_urb}
depicts this difference in landscape by comparing the distribution of
community types in Texas and Wisconsin in 2010 (bar widths reflect the
relative quantity of districts in Texas and Wisconsin). While both
states are majority-rural, the non-rural part of Texas is comparatively
urbanized, whereas more than 90\% of Wisconsin districts are non-urban.

Returning to Table \ref{tbl:markov}, we see that, as in HKR, the
``stickiest'' community type is Rural -- over 60\% of Rural teachers
remain Rural in both papers, and even fewer Rural Wisconsin teachers end
up in a big city than is the case for Texas. This may reflect the
similarity in prevalence of rural districts in the two states and a
natural similarity in preferences of rural teachers and districts.
Lastly, we also find that the community type transition patterns of
younger teachers as compared to all teachers are broadly similar.

\begin{sidewaystable}[htbp]
\centering
\begin{tabular}{lccccccc}
  \hline
 & \multicolumn{3}{c}{Men by Experience Class} & \multicolumn{3}{c}{Women by Experience Class} & \multirow{2}{*}{\parbox{0.1\linewidth}{All Teachers 0-9 Years}}\\ \cline{2-4} \cline{5-7}
 & 1-3 years & 4-6 years & 7-11 years & 1-3 years & 4-6 years & 7-11 years &  \\ 
  \hline
Base year salary (log) & -0.001 & 0.015 & 0.036 & 0.001 & 0.022 & 0.010 & 0.009 \\ 
   & (0.009) & (0.011) & (0.015) & (0.005) & (0.007) & (0.010) & (0.003) \\ 
  Adjusted salary\textsuperscript{a} (log) & 0.011 & 0.003 & 0.024 & 0.002 & 0.014 & 0.015 & 0.008 \\ 
   & (0.007) & (0.009) & (0.012) & (0.004) & (0.006) & (0.008) & (0.003) \\ 
  Percent proficient & 4.2\% & 3.0\% & 2.5\% & 6.3\% & 5.7\% & 5.3\% & 5.4\% \\ 
   & (0.7\%) & (0.8\%) & (1.0\%) & (0.4\%) & (0.5\%) & (0.6\%) & (0.2\%) \\ 
  Percent Hispanic & -1.4\% & -0.3\% & -0.2\% & -1.7\% & -1.6\% & -1.1\% & -1.4\% \\ 
   & (0.3\%) & (0.4\%) & (0.5\%) & (0.2\%) & (0.2\%) & (0.3\%) & (0.1\%) \\ 
  Percent black & -5.4\% & -2.1\% & -3.8\% & -8.6\% & -6.8\% & -6.9\% & -7.0\% \\ 
   & (1.0\%) & (1.1\%) & (1.2\%) & (0.6\%) & (0.7\%) & (0.8\%) & (0.3\%) \\ 
  Percent subsidized lunch & -7.4\% & -3.7\% & -4.4\% & -9.5\% & -7.0\% & -7.6\% & -7.9\% \\ 
   & (1.1\%) & (1.4\%) & (1.7\%) & (0.6\%) & (0.9\%) & (1.0\%) & (0.4\%) \\ 
   \hline
\multicolumn{8}{c}{\parbox{0.85\linewidth}{\scriptsize{Note: a. Adjusted salary is the residual of log salary by district and experience level on 12 regional indicators, three urbanicity indicators, and the district percentages proficient on the WKCE exam, black, Hispanic, and low income.}}}\\
\end{tabular}
\caption{Average Change in Salary and District Student Characteristics (and Standard Deviations) for Teachers Changing Districts, by Gender and Experience} 
\label{tbl:change_by_ge}
\end{sidewaystable}

Table \ref{tbl:change_by_ge} replicates Table 3 of HKR, and again
confirms its most important insights. Raw salary differentials predict
teacher mobility, but the average pay differential is not on average
very large -- only about \$325, or 1.7\% higher than the
counterfactually expected wage that would have obtained had the
district-switching teacher remained in their current district\footnote{There
  are 777 teachers in the data who skipped one or more years before
  reappearing at different school or district (perhaps representing
  leaves of absence for retraining or re-adjustment). For such teachers,
  the counterfactual subsequent experience and reference curves are
  taken from their next year in the data, rather than from simply
  incrementing their experience by one.}. This premium increases with
age for both male and female teachers.

\begin{figure}[htbp]
\centering
\includegraphics{figures/potential_premium-1.pdf}
\caption{\label{fig:premia}How Much Do Teachers Stand to Gain from
Changing Districts throughout Their Careers?}
\end{figure}

One potential explanation of the weakness of the wage results is that
there simply is not sufficient heterogeneity among available contracts
to generate mobility incentives. Figure \ref{fig:premia} demonstrates
that this is not likely the case. No matter their current experience or
certification level, a teacher in a district paying the 25th percentile
of wages for that experience-certification cell would gain on average
17\% by changing to a district at the 75th percentile. Especially for
younger teachers, this potential gain would accumulate annually to
become a hefty sum over the course of the career -- discounting the
average annual gain for Master's-certified teachers at 6\% and adding
over 20 years, this means roughly \$100,000 is on the table; results are
more dramatic for teachers at districts further in the tails of the wage
distribution.

Attempting to isolate the influence of district characteristics on wage
effects, HKR suggest comparing the differential leverage of residual
wages (the residuals being the unexplained part of a Mincer-type
regression) to get a more focused estimate of the association between
wages and mobility\footnote{HKR mention they failed to adjust the
  standard errors associated with the adjusted wage differentials to
  account for the fact that they involve residuals from a regression. We
  explored accounting for this by bootstrapping the regression through
  resampling teachers and recalculating residuals, but little changes as
  a result, so we present the naive standard errors for simplicity.}. We
run a similar regression, but evaluate separate regressions not just for
each level of experience, but also for each certification track. This
leads to a boost in the overall fraction of explained variance from 60\%
cited by HKR to 87\% here; as in HKR, other included covariates are
consistently significant, suggesting their strong independent
correlation with salary levels.

Unlike HKR, we find the demographic-independent wage differentials to be
no more important than the uncontrolled raw wages, with the predicted
wage improvement amounting to 0.8\%. In further contrast to HKR, we find
a positive relationship between experience and residual wage
differentials, with mid-career district switchers experiencing roughly
1.4\% higher wages upon arrival to their new employer, by contrast to
the null relationship for probationary teachers. This pattern is
consistent across the dimension of certification which was ignored by
HKR, suggesting the opposite result cannot be attributed to bias
introduced by movement patterns of Bachelor's- vs.~Master's-certified
instructors.

Student demographic differentials are very important for predicting
teacher turnover, a finding which held in Texas as it does in Wisconsin.
Most distinguished in all experience classes and for both genders are
changes in measures of student performance, student poverty and the
percetnage of black students -- district switchers end up at schools
with 5\% more students at grade level overall, an effect which is
stronger for female teachers and for young teachers. They also end up on
average with about 8\% fewer students (school-wide) eligible for
subsidized lunch and 7\% fewer black students. While this finding would
need to be bolstered with experimental or quasi-experimental evidence,
it hints at the potentially limited scope of teacher labor market
policies intended to ameliorate teacher supply problems in hard-to-serve
districts -- schools can much more easily exert influence over their
compensation policies than they can dictate their student bodies, but
the latter appears more efficacious (see Fulbeck
\protect\hyperlink{ref-fulbeck}{2014} and Glazerman et al.
(\protect\hyperlink{ref-glazerman}{2013})).

\begin{sidewaystable}[htbp]
\centering
\begin{tabular}{lccccccc}
  \hline
 & \multicolumn{3}{c}{Men by Experience Class} & \multicolumn{3}{c}{Women by Experience Class} & \multirow{2}{*}{\parbox{0.1\linewidth}{All Teachers 0-9 Years}}\\ \cline{2-4} \cline{5-7}
 & 1-3 years & 4-6 years & 7-11 years & 1-3 years & 4-6 years & 7-11 years &  \\ 
  \hline
Base year salary (log) & 0.020 & 0.012 & -0.008 & -0.019 & 0.009 & 0.009 & -0.002 \\ 
   & (0.015) & (0.023) & (0.033) & (0.012) & (0.014) & (0.022) & (0.007) \\ 
  Adjusted salary (log) & 0.016 & -0.032 & -0.002 & -0.005 & 0.002 & 0.014 & -0.000 \\ 
   & (0.011) & (0.017) & (0.031) & (0.010) & (0.010) & (0.015) & (0.005) \\ 
  Percent proficient & 2.1\% & 3.1\% & 1.2\% & 4.3\% & 2.9\% & 4.7\% & 3.5\% \\ 
   & (1.0\%) & (1.2\%) & (1.5\%) & (0.6\%) & (1.0\%) & (1.3\%) & (0.4\%) \\ 
  Percent Hispanic & -0.7\% & 0.3\% & -0.3\% & -1.3\% & -1.2\% & -1.1\% & -1.0\% \\ 
   & (0.5\%) & (0.6\%) & (0.8\%) & (0.3\%) & (0.4\%) & (0.6\%) & (0.2\%) \\ 
  Percent black & -1.7\% & -0.4\% & -2.1\% & -4.3\% & -2.8\% & -5.0\% & -3.3\% \\ 
   & (1.3\%) & (1.2\%) & (1.6\%) & (0.9\%) & (1.3\%) & (1.8\%) & (0.5\%) \\ 
  Percent subsidized lunch & -5.9\% & -5.6\% & -2.6\% & -7.2\% & -3.8\% & -6.1\% & -5.9\% \\ 
   & (1.6\%) & (2.0\%) & (3.0\%) & (1.0\%) & (1.5\%) & (1.9\%) & (0.6\%) \\ 
   \hline
\end{tabular}
\caption{Average Change in Salary and District Student Characteristics (and Standard Deviations) for Teachers Changing to a district more than 50 Miles Away, by Gender and Experience} 
\label{tbl:change_far_by_ge}
\end{sidewaystable}

\subsection{Long-Distance Moves}\label{long-distance-moves}

One major aspect of teacher mobility glossed over by HKR is geographic
separation. A wide variety of frictions may be geospatially-related or
-generated -- social and professional networks tend to be concentrated
locally; there are typically substantial fixed costs involved in moving
(real estate closing fees, moving expenses, etc.); preferences may
depend on climate/geography; and so on. As a first pass at exploring how
long-distance moves may differ in nature from those over short
distances, we reproduce in Table \ref{tbl:change_far_by_ge} the analysis
of Table \ref{tbl:change_by_ge} for only those moves where the distance
between the origin and destination school exceeded 50 miles (a distance
deemed sufficient to likely entail a physical move rather than simply an
adjusted commute).

The preeminent distinction of long-distance moves is moderation -- all
average demographic differentials moderate towards zero, suggesting a
diminution of the importance of these aspects in this population. The
noteworthy exception to this trend is among probationary teachers --
young males experience wage increases in an uprooting move, while young
females experience declines for long moves. More detailed data would be
needed to explore the mechanism at work behind this observation (in
particular, none of the differences -- male vs.~female or short-
vs.~long-distance moves -- have \(p\) values below .05), but one
explanation is a higher willingness among bachelors to change scenery
completely, while younger women may tend to be married and moving with
their partners. In any case, the overall importance of wages in
long-distance moves is close to zero, suggesting wage differentials are
either of secondary or tertiary concern in the associated decision
processes, or that there is unsufficient heterogeneity in wages at such
distances to generate enough moves so motivated, though the case of
young male teachers does weaken the latter explanation.

\subsection{Supply and Demand for Subject
Specialists}\label{supply-and-demand-for-subject-specialists}

Another source of heterogeneity about which HKR have little to say is
subject specialty. While it is true that all teachers on a given
contract are typically paid independently of the subject they teach,
teaching a hard-to-staff subject should lead to more bargaining power in
the labor market (as such teachers are less easily replaced), so we
would expect such teachers to transition to more attractive positions
upon moving. Fully accounting for the demand side of labor markets would
bestow higher confidence in results which ultimately depend on the
strategic interaction of the two sides.

We are aided in trying to explore this aspect of the teacher labor
market by the public availability of annual technical reports from DPI
about various aggregate indicators for the health of supply and demand
for educators in Wisconsin (the last published edition is Fischer,
Swanger, and Skoning \protect\hyperlink{ref-fischer}{2009}). In addition
to providing counts for the number of educators graduating from the
in-state education programs broken down by subject area, the report uses
a survey distributed to district administrators to give a score (based,
for example, on the market tightness -- applications per vacancy) in
each Cooperative Educational Service Agency (CESA, the administrative
unit for districts between the school district and DPI) rating the need
for educators in various subject areas, including those in our study
sample, Math, Reading, and Elementary.

\begin{sidewaystable}[htbp]
\centering
\begin{tabular}{lcccccccc}
  \hline
 & \multicolumn{2}{c}{1-3 years} & \multicolumn{2}{c}{4-6 years} & \multicolumn{2}{c}{7-11 years} & \multicolumn{2}{c}{All}\\ \cline{2-3} \cline{4-5} \cline{6-7} \cline{8-9}
 & Non-Math & Math & Non-Math & Math & Non-Math & Math & Non-Math & Math \\ 
  \hline
Base year salary (log) & -0.069 & 0.031 & 0.007 & 0.015 & 0.025 & -0.005 & 0.001 & 0.011 \\ 
   & (0.030) & (0.042) & (0.019) & (0.053) & (0.015) & (0.031) & (0.012) & (0.022) \\ 
  Adjusted salary (log) & -0.008 & 0.068 & 0.031 & 0.047 & 0.034 & 0.019 & 0.025 & 0.041 \\ 
   & (0.024) & (0.024) & (0.014) & (0.047) & (0.013) & (0.017) & (0.009) & (0.015) \\ 
  Percent proficient & 10.8\% & 8.5\% & 6.2\% & 1.4\% & 4.4\% & 2.2\% & 6.2\% & 3.7\% \\ 
   & (2.0\%) & (3.4\%) & (1.3\%) & (2.5\%) & (1.0\%) & (1.9\%) & (0.8\%) & (1.5\%) \\ 
  Percent Hispanic & -4.2\% & 1.4\% & -2.1\% & -1.1\% & -0.8\% & -1.4\% & -1.9\% & -0.6\% \\ 
   & (1.0\%) & (1.7\%) & (0.6\%) & (1.5\%) & (0.5\%) & (1.3\%) & (0.4\%) & (0.9\%) \\ 
  Percent black & -15.0\% & -11.8\% & -6.6\% & 0.4\% & -5.1\% & -4.2\% & -7.5\% & -4.9\% \\ 
   & (3.2\%) & (5.3\%) & (1.8\%) & (3.3\%) & (1.2\%) & (2.5\%) & (1.1\%) & (2.1\%) \\ 
  Percent subsidized lunch & -17.6\% & -7.3\% & -7.9\% & -2.9\% & -4.9\% & -4.8\% & -8.4\% & -4.9\% \\ 
   & (3.3\%) & (5.7\%) & (2.3\%) & (5.2\%) & (1.6\%) & (3.4\%) & (1.3\%) & (2.6\%) \\ 
   \hline
\end{tabular}
\caption{Average Change in Salary and District Student Characteristics (and Standard Deviations) for Teachers with Master's Degrees Changing Districts, by Subject Area and Experience} 
\label{tbl:change_area_by_ge}
\end{sidewaystable}

Both Reading and Elementary are chronically over-supplied throughout the
state, whereas the demand for math teachers varies considerably. In a
given year, the market tightness for the former two subjects is roughly
twice that in Math (e.g., it was 67.43 for Elementary, 28.65 for
English/Speech/Theater/Journalism, and 24.22 for Mathematics). As a
result, we expect to see some heterogeneity in labor market success of
specialists in Math as compared to the other teachers in our sample.
Table \ref{tbl:change_area_by_ge} explores some of the basic insights on
subject matter heterogeneity. To mitigate the potential for degree
holdings to skew results, we focus on Master's holders and obfuscate
gender differences for brevity. Actually, there is little in this table
to support the hypothesis that math teachers are given a substantial
advantage in the labor market -- math teachers earn more (both in
nominal and beyond-demographic pay), but this result is not significant.
Further, English teachers are advantaged in ending up at less
economically disadvantaged and higher-performing districts.

\begin{table}[htbp]
\centering
\begin{tabular}{lp{.135\textwidth}p{.135\textwidth}p{.135\textwidth}p{.135\textwidth}}
  \hline
 & \multicolumn{2}{c}{\multirow{2}{*}{\parbox{0.2\linewidth}{District Average Characteristics}}} & \multicolumn{2}{c}{\multirow{2}{*}{\parbox{0.2\linewidth}{Campus Average Characteristics}}}\\
 & & & & \\ \cline{2-5}
 & Large Urban to Suburban & Suburban to Suburban & Large Urban to Suburban & Suburban to Suburban \\
  \hline
Base year salary (log) & -0.056 & 0.018 & --- & --- \\ 
   & (0.012) & (0.007) &  &  \\ 
  Adjusted salary (log) & -0.004 & 0.011 & --- & --- \\ 
   & (0.005) & (0.006) &  &  \\ 
Average Student Characteristics & & & & \\
  \quad Percent proficient & 37.9\% & 0.9\% & 35.1\% & 0.1\% \\ 
   & (0.6\%) & (0.4\%) & (1.2\%) & (0.6\%) \\ 
  \quad Percent Hispanic & -11.3\% & -0.6\% & -7.3\% & -0.4\% \\ 
   & (0.4\%) & (0.2\%) & (1.3\%) & (0.2\%) \\ 
  \quad Percent black & -56.9\% & -0.6\% & -59.7\% & -0.5\% \\ 
   & (0.8\%) & (0.3\%) & (1.8\%) & (0.4\%) \\ 
  \quad Percent subsidized lunch & -55.7\% & -1.7\% & -61.1\% & -1.6\% \\ 
   & (1.2\%) & (0.5\%) & (1.3\%) & (0.7\%) \\ 
   \hline
\end{tabular}
\caption{Average Change in Salary and in District and Campus Student Characteristics (and Standard Deviations) for Teachers with 1-10 Years of Experience Who Change Districts, by Community Type of Origin and Destination District} 
\label{tbl:change_by_urb}
\end{table}

Table \ref{tbl:change_by_urb}, which parallels Table 4 of HKR, again
uncovers a labor market functioning similar to that in Texas. In
particular, while HKR find Large Urban - Suburban district switchers
penalize themselves in pay but are rewarded in demographic-adjusted pay,
Wisconsin teachers lose out on both measures when leaving Large Urban
districts, albeit the residual pay penalty is much lower than that of
nominal pay. This difference does not appear to be attributable to HKR's
exclusion of certification as a conditioning variable, as the pattern
here differs insignificantly by degree.

The other results of HKR are confirmed in even more dramatic fashion.
There is strong evidence of selection on the student performance metric,
which does vary quite widely in suburban districts. Teachers leaving
Milwaukee tend to end up at districts with 38\% more students deemed to
be at grade level on the state standardized test. On the other hand,
teachers leaving Large Urban districts (i.e, Milwaukee) for the suburbs
experience a precipitous drop of 57\% black students and 56\% subsidized
lunch eligibility. This is practically a tautological result, as the
student demographics outside of urban areas in Wisconsin are pretty
uniformly non-minority -- about 90\% of suburban districts have fewer
than 10\% black students, and about 60\% have fewer than 2\% black
students, whereas Milwaukee is about 60\% black. Similarly, teachers
leaving Milwaukee for the suburbs have little choice but to end up in a
district with far fewer economically disadvantaged students -- whereas
73\% of Milwaukee students are eligible, the median percentage in
suburban schools is 12\%.

The direction of these effects are preserved among suburban-to-suburban
moves, suggesting the importance of these factors even in areas where
there is a wider array demographically of destination districts. We also
find evidence of selection into economically better-off districts among
suburban switchers, but the magnitude of this difference is attenuated
with respect to that reported by HKR. We do not find patterns of
selection on student performance as strongly as was found in HKR. This
may be a reflection of the crudeness of the proficiency measure as
compared to the more variable raw scale score measures used by HKR.
Lastly, we confirm the finding of HKR that there does not appear to be
evidence that teachers are able to select into the more desirable
schools within their target districts -- the differences in campus-level
characteristics are almost identical to the differences in
district-level characteristics. This is likely a reflection of
supply-side constraints, as the choicest appointments in a district may
be awarded to long-serving serving teachers (promotion from within), as
well as suburban districts perhaps having only a small number of schools
at which to teach a given grade level/subject.

\begin{table}[htbp]
\centering
\begin{tabular}{lp{.1\textwidth}p{.1\textwidth}p{.1\textwidth}p{.1\textwidth}}
  \hline
 & \multicolumn{2}{c}{Between District Moves} & \multicolumn{2}{c}{Within District Moves}\\ \cline{2-5}
 & Black Teachers & Hispanic Teachers & Black Teachers & Hispanic Teachers \\
  \hline
Percent proficient & 10.7\% & 8.0\% & 2.7\% & 2.2\% \\ 
   & (3.4\%) & (5.6\%) & (0.9\%) & (1.3\%) \\ 
  Percent Hispanic & 3.2\% & -14.8\% & 1.0\% & -7.7\% \\ 
   & (1.4\%) & (7.3\%) & (0.9\%) & (2.3\%) \\ 
  Percent black & -21.1\% & -0.6\% & -2.1\% & -0.3\% \\ 
   & (5.0\%) & (5.0\%) & (1.4\%) & (2.0\%) \\ 
  Percent subsidized lunch & -19.1\% & -15.5\% & -3.5\% & -4.7\% \\ 
   & (7.7\%) & (6.7\%) & (0.8\%) & (1.3\%) \\ 
  Number of teachers & 81 & 37 & 638 & 228 \\ 
   \hline
\end{tabular}
\caption{Average Change in District and Campus Student Characteristics (and Standard Deviations) for Black and Hispanic Teachers with 1-10 Years of Experience who Change Campuses} 
\label{tbl:change_by_eth}
\end{table}

HKR examine the state of Texas, which features substantially more ethnic
heterogeneity than does Wisconsin. As a result, they are better-equipped
to identify heterogeneity in preferences by teacher ethnicity. In
Wisconsin, however, only 2,372 of the 49,325 teachers are non-white, so
our results are underpowered relative to HKR. For completeness, Table
\ref{tbl:change_by_eth} presents these results, which parallel HKR Table
5. Given how few observations we have of black or Hispanic teachers
switching districts, we eschew any temptation to interpret these
results. Only black switchers within districts provide enough records to
interpret meaningfully. In Wisconsin, we find that, in contrast to white
within-district switchers, black teachers tend to migrate to
economically better-off and higher-performing schools (white teachers
also select on percentage of black students). This could simply be a
reflection of differences in initial district choice by black vis-à-vis
white teachers -- the median proficiency at a black teacher's first
district is 36\%, compared to 64\% for white teachers (71\% and 22\% for
reduced lunch eligibility, respectively).

\begin{table}[htbp]
\centering
\begin{tabular}{p{.3\textwidth}p{.15\textwidth}p{.15\textwidth}p{.15\textwidth}}
  \hline
Quartile of Distribution & Probability Teachers Move to New School within District & Probability Teachers Move to New District & Probability Teachers Exit Public Schools \\ 
  \hline
Residual salary & & & \\
\quad Highest & --- & 1.5\% & 4.1\% \\ 
  \quad 3rd & --- & 1.8\% & 5.0\% \\ 
  \quad 2nd & --- & 1.8\% & 4.9\% \\ 
  \quad Lowest & --- & 1.9\% & 4.1\% \\ 
Percent proficient & & & \\
  \quad Highest & 4.5\% & 1.9\% & 4.2\% \\ 
  \quad 3rd & 4.6\% & 2.3\% & 4.2\% \\ 
  \quad 2nd & 5.2\% & 1.7\% & 4.4\% \\ 
  \quad Lowest & 6.1\% & 2.1\% & 4.6\% \\ 
Percent eligible for reduced-price lunch & & & \\
  \quad Highest & 7.1\% & 2.1\% & 5.3\% \\ 
  \quad 3rd & 5.6\% & 1.7\% & 3.8\% \\ 
  \quad 2nd & 4.1\% & 2.0\% & 3.9\% \\ 
  \quad Lowest & 3.6\% & 2.2\% & 4.4\% \\ 
Percent Black & & & \\
  \quad Highest & 7.3\% & 2.1\% & 5.9\% \\ 
  \quad 3rd & 4.9\% & 1.6\% & 4.2\% \\ 
  \quad 2nd & 4.7\% & 1.9\% & 3.8\% \\ 
  \quad Lowest & 3.4\% & 2.4\% & 3.4\% \\ 
Percent Hispanic & & & \\
  \quad Highest & 7.6\% & 1.8\% & 5.5\% \\ 
  \quad 3rd & 4.4\% & 2.0\% & 4.1\% \\ 
  \quad 2nd & 4.3\% & 2.0\% & 4.0\% \\ 
  \quad Lowest & 4.0\% & 2.3\% & 3.7\% \\ 
   \hline
\end{tabular}
\caption{School Average Transition Rates by Distribution of Residual Teacher Salary and Student Demographic Characteristics (data weighted by number of teachers in school)} 
\label{tbl:change_by_quartile}
\end{table}

To the end of examining heterogeneity in the impact of school and
district characteristic differentials on teacher mobility, HKR present
their Table 6, which breaks down the three exit rates for each
(weighted) quartile of the covariate distribution. We replicate that
analysis here in Table \ref{tbl:change_by_quartile}. Saliently, our
results for the correlation of school characteristics for
within-district movers are qualitatively identical to those found in
Texas and similar in magnitude, which gives a stronger indication that
we have identified some fundamental nonpecuniary mechanisms driving
sorting among schools in a district.

Differences with respect to the results in Texas begin to emerge for the
other destinations of school leavers (other districts and other
professions). As noted in Table \ref{tbl:move_by_exp}, overall rates of
switching districts are quite low compared to Texas and national
averages; conditional on this, the patterns of movement by quartile of
residual salary exhibit a similar pattern to that in Texas, with
teachers in the lowest quartile about 28\% more likely to change
districts than teachers in the highest residual pay quartile. By
contrast to HKR, however, who found the opposite association, we find
the same trend (at attenuated magnitudes) with respect to leaving
Wisconsin public schools, suggesting salary considerations are also
important for teachers considering options outside of public school
teaching (or in other states).

We also find fairly strong patterns in quitting associated with
subsidized lunch eligibility and with the ethnic makeup of schools, with
teachers at the most economically advantaged schools 8\% less likely to
exit teaching; similar numbers obtain for both the quantity of black and
of Hispanic students. For teachers moving within districts, we observe
similar patterns.

\subsection{Regression Results}\label{regression-results}

\begin{sidewaystable}
\begin{center}
\begin{tabular}{l c c c c c }
\hline
 & \multicolumn{4}{c}{Teacher Experience} \\ \cline{2-6}
 & 1-3 years & 4-6 years & 7-11 years & 12-30 years & >30 years \\
\hline
First year base salary (log)                & $0.03$        & $-0.09^{**}$  & $-0.04^{*}$   & $0.01$       & $-0.12$     \\
                                            & $(0.03)$      & $(0.03)$      & $(0.02)$      & $(0.01)$     & $(0.06)$    \\
First year base salary (log) * female       & $-0.07^{*}$   & $0.09^{**}$   & $0.02$        & $-0.02$      & $0.12^{*}$  \\
                                            & $(0.03)$      & $(0.03)$      & $(0.02)$      & $(0.01)$     & $(0.06)$    \\
Campus average student characteristics      &               &               &               &              &             \\
\quad Percent proficient                    & $-0.10^{*}$   & $0.03$        & $-0.02$       & $0.00$       & $-0.07$     \\
                                            & $(0.05)$      & $(0.04)$      & $(0.02)$      & $(0.01)$     & $(0.06)$    \\
\quad Percent eligible for subsidized lunch & $-0.09^{**}$  & $-0.07^{**}$  & $-0.07^{***}$ & $-0.01$      & $0.07$      \\
                                            & $(0.03)$      & $(0.03)$      & $(0.02)$      & $(0.01)$     & $(0.04)$    \\
\quad Percent Black                         & $0.04$        & $0.22^{***}$  & $0.15^{***}$  & $0.09^{***}$ & $0.14^{*}$  \\
                                            & $(0.04)$      & $(0.04)$      & $(0.03)$      & $(0.02)$     & $(0.07)$    \\
\quad Percent Hispanic                      & $0.13^{*}$    & $0.16^{***}$  & $0.04$        & $-0.04^{*}$  & $-0.19^{*}$ \\
                                            & $(0.06)$      & $(0.05)$      & $(0.03)$      & $(0.02)$     & $(0.08)$    \\
Interactions                                &               &               &               &              &             \\
\quad Black * percent Black                 & $-0.21^{**}$  & $-0.13$       & $-0.03$       & $-0.03$      & $-0.21$     \\
                                            & $(0.08)$      & $(0.07)$      & $(0.05)$      & $(0.04)$     & $(0.12)$    \\
\quad Hispanic * percent Black              & $-0.19^{***}$ & $-0.19^{***}$ & $-0.12^{*}$   & $-0.12^{**}$ & $0.02$      \\
                                            & $(0.06)$      & $(0.05)$      & $(0.05)$      & $(0.04)$     & $(0.34)$    \\
\quad Black * percent Hispanic              & $0.14$        & $-0.23$       & $-0.14$       & $0.03$       & $-0.53$     \\
                                            & $(0.25)$      & $(0.23)$      & $(0.14)$      & $(0.11)$     & $(0.45)$    \\
\quad Hispanic * percent Hispanic           & $0.14$        & $-0.07$       & $-0.21$       & $0.26$       & $0.61$      \\
                                            & $(0.28)$      & $(0.23)$      & $(0.20)$      & $(0.19)$     & $(1.07)$    \\
\hline
Observations                                & 33,108         & 30,244         & 43,509         & 98,753        & 15,217       \\
\hline
\multicolumn{6}{l}{\scriptsize{$^{***}p<0.001$, $^{**}p<0.01$, $^*p<0.05$}}
\end{tabular}
\caption{Estimated Effects of Starting Teacher Salary and Student Demographic Characteristics on the Probability that Teachers Leave School Districts, by Experience (linear probability models; Huber-White standard  errors in parentheses)}
\label{tbl:reg_lpm}
\end{center}
\end{sidewaystable}

Having identified some key patterns in moments of the data, we now move
on to try and separate the confounding effects of each of these and
other factors in affecting teacher turnover with the aim of identifying
more fundamentally the association between salient district and school
characteristics on teacher turnover. Table \ref{tbl:reg_lpm} provides
the main coefficients of interest from a simple linear probability
regression model predicting leaving a district (i.e., either switching
districts or exiting teaching); this corresponds to HKR Table 7.

By contrast to the strength implied in earlier results, the importance
of student achievement has dwindled in the regression specification, and
only comes out as independently significant for probationary teachers.
The same goes for base salary differentials -- in contrast to HKR, the
evidence we find in favor of an independent influence of salary on
turnover rates is sparse and concentrated among young teachers\footnote{HKR
  also mention results not printed in their paper suggesting a paucity
  of evidence suggesting class size is an important factor in teacher
  turnover decisions; we give tepid support to this statement, as class
  size does indeed appear to be related to turnover, but somewhat weakly
  and only for younger teachers.}. This does not appear to be due to
imprecision -- the magnitude of HKR's standard errors follows closely
those found for the Wisconsin data, despite our smaller sample sizes.

HKR also found little independent evidence in favor of student economic
status factoring in to teachers' mobility decisions, but we find fairly
consistent support for the importance of subsidized lunch eligibility
prevalence. As mentioned above, it is possible that the crude nature of
the proficiency measure is only weakly identified, and that some of the
unaccounted for part of student performance is being captured in other
coefficients, especially subsidized lunch eligibility and student race.
Even more compelling would be to associate student performance (and
other school/district-level characteristics) more finely with the set of
students actually faced by a given teacher.

The results in HKR about the differential effects of student body makeup
are largely similar to those we find in Wisconsin. White and nonwhite
teachers have opposite and significant correlations between the quantity
of minority students in their origin district and their likelihood of
leaving it. These differential results tend to modulate towards zero
with experience, regardless of teacher or student race category, and
suggest a degree of assortative matching on ethnicity among districts in
Wisconsin (though the patterns for whites differ sharply from those of
nonwhites, the patterns for black and Hispanic teachers are hard to
distinguish).

\begin{sidewaystable}
\begin{center}
\begin{tabular}{l c c c c c }
\hline
 & \multicolumn{4}{c}{Teacher Experience} \\ \cline{2-6}
 & 1-3 years & 4-6 years & 7-11 years & 12-30 years & >30 years \\
\hline
First year base salary (log)                & $0.01$       & $-0.13^{***}$ & $-0.05^{**}$ & $0.00$       & $-0.17^{**}$ \\
                                            & $(0.03)$     & $(0.04)$      & $(0.02)$     & $(0.01)$     & $(0.07)$     \\
First year base salary (log) * female       & $-0.07^{*}$  & $0.09^{**}$   & $0.03$       & $-0.02$      & $0.13^{*}$   \\
                                            & $(0.03)$     & $(0.03)$      & $(0.02)$     & $(0.01)$     & $(0.06)$     \\
Campus average student characteristics      &              &               &              &              &              \\
\quad Percent proficient                    & $-0.12$      & $-0.04$       & $-0.10^{*}$  & $0.01$       & $-0.09$      \\
                                            & $(0.08)$     & $(0.06)$      & $(0.04)$     & $(0.02)$     & $(0.10)$     \\
\quad Percent eligible for subsidized lunch & $-0.07$      & $-0.05$       & $-0.03$      & $-0.02$      & $-0.21^{*}$  \\
                                            & $(0.08)$     & $(0.06)$      & $(0.04)$     & $(0.02)$     & $(0.11)$     \\
\quad Percent Black                         & $0.29$       & $0.46$        & $0.54^{**}$  & $0.03$       & $0.38$       \\
                                            & $(0.28)$     & $(0.26)$      & $(0.18)$     & $(0.10)$     & $(0.48)$     \\
\quad Percent Hispanic                      & $0.04$       & $0.04$        & $-0.23^{*}$  & $-0.07$      & $0.13$       \\
                                            & $(0.18)$     & $(0.15)$      & $(0.09)$     & $(0.05)$     & $(0.25)$     \\
Interactions                                &              &               &              &              &              \\
\quad Black * percent Black                 & $-0.19^{*}$  & $-0.15^{*}$   & $-0.06$      & $-0.04$      & $-0.17$      \\
                                            & $(0.08)$     & $(0.07)$      & $(0.05)$     & $(0.04)$     & $(0.12)$     \\
\quad Hispanic * percent Black              & $-0.17^{**}$ & $-0.18^{***}$ & $-0.12^{*}$  & $-0.11^{**}$ & $-0.09$      \\
                                            & $(0.06)$     & $(0.05)$      & $(0.05)$     & $(0.04)$     & $(0.33)$     \\
\quad Black * percent Hispanic              & $0.06$       & $-0.15$       & $0.02$       & $0.09$       & $-0.73$      \\
                                            & $(0.26)$     & $(0.25)$      & $(0.15)$     & $(0.11)$     & $(0.48)$     \\
\quad Hispanic * percent Hispanic           & $0.14$       & $-0.04$       & $-0.12$      & $0.26$       & $0.26$       \\
                                            & $(0.28)$     & $(0.23)$      & $(0.21)$     & $(0.20)$     & $(1.10)$     \\
\hline
Observations                                & 33,108        & 30,244         & 43,509        & 98,753        & 15,217        \\
\hline
\multicolumn{6}{l}{\scriptsize{$^{***}p<0.001$, $^{**}p<0.01$, $^*p<0.05$}}
\end{tabular}
\caption{Estimated Effects of Starting Teacher Salary and Student Demographic Characteristics on the Probability that Teachers Leave School Districts with District Fixed Effects, by Experience (linear probability models; Huber-White standard errors in parentheses)}
\label{tbl:reg_lpm_fe}
\end{center}
\end{sidewaystable}

To account in a rudimentary way for district-specific hiring policies,
HKR move on to their Table 8 which repeats Table 7 (our Table
\ref{tbl:reg_lpm}) with district fixed effects. HKR note that the
patterns in responsiveness to wages are the same, though attenuated;
that coefficients involving student ethnicity are qualitatively
unaffected; and that schools with high achievement continue to exhibit
lower propensities for turnover. Our results, presented in Table
\ref{tbl:reg_lpm_fe}, are similar in that they closely resemble the
results without fixed effects, but with noted attenuation and weaker
precision.

The most notable difference relative to Table \ref{tbl:reg_lpm} is the
general weakening of results regarding the importance of student
characteristics for white teachers. While partially attributable to a
decline in precision, this adjustment suggests much of the discovered
correlation between student characteristics and exit probability for
white teachers can be chalked up to district-to-district heterogeneity
in preferences or hiring policies.

\begin{sidewaystable}
\begin{center}
\begin{tabular}{l c c c c }
\hline
 & \multicolumn{4}{c}{Teacher Experience} \\ \cline{2-5}
 & 1-3 years & 4-6 years & 7-11 years & 12-30 years \\
\hline
I. Switch Districts                             &               &               &               &              \\
\quad First year base salary (log)           & $0.66$        & $-0.27$       & $-0.81$       & $1.36$       \\
                                                & $(0.53)$      & $(0.74)$      & $(0.71)$      & $(0.91)$     \\
\quad First year base salary (log) * female  & $-1.12^{*}$   & $-0.10$       & $-0.36$       & $-1.83^{*}$  \\
                                                & $(0.52)$      & $(0.73)$      & $(0.73)$      & $(0.91)$     \\
\quad Percent proficient                     & $-1.38^{*}$   & $0.16$        & $-1.05$       & $1.20$       \\
                                                & $(0.60)$      & $(0.80)$      & $(0.89)$      & $(1.03)$     \\
\quad Percent eligible for subsidized lunch  & $-0.57$       & $-0.91$       & $-1.98^{**}$  & $-0.62$      \\
                                                & $(0.40)$      & $(0.57)$      & $(0.64)$      & $(0.74)$     \\
\quad Percent Nonwhite                       & $0.15$        & $2.11^{***}$  & $2.87^{***}$  & $3.19^{***}$ \\
                                                & $(0.40)$      & $(0.54)$      & $(0.58)$      & $(0.66)$     \\
\quad Nonwhite * percent Nonwhite            & $-2.45^{***}$ & $-3.68^{***}$ & $-3.08^{**}$  & $0.11$       \\
                                                & $(0.56)$      & $(0.87)$      & $(1.04)$      & $(1.29)$     \\
II. Exit Teaching                               &               &               &               &              \\
\quad First year base salary (log)           & $0.27$        & $-2.29^{***}$ & $-0.76$       & $0.05$       \\
                                                & $(0.52)$      & $(0.57)$      & $(0.64)$      & $(0.44)$     \\
\quad First year base salary (log) * female  & $-0.63$       & $2.43^{***}$  & $0.73$        & $-0.32$      \\
                                                & $(0.51)$      & $(0.61)$      & $(0.65)$      & $(0.44)$     \\
\quad Percent proficient                     & $-0.46$       & $0.42$        & $0.02$        & $0.12$       \\
                                                & $(0.57)$      & $(0.68)$      & $(0.71)$      & $(0.45)$     \\
\quad Percent eligible for subsidized lunch  & $-0.88^{*}$   & $-1.50^{**}$  & $-2.31^{***}$ & $-0.57$      \\
                                                & $(0.39)$      & $(0.49)$      & $(0.53)$      & $(0.31)$     \\
\quad Percent Nonwhite                       & $1.24^{***}$  & $3.17^{***}$  & $2.36^{***}$  & $0.90^{**}$  \\
                                                & $(0.35)$      & $(0.42)$      & $(0.46)$      & $(0.29)$     \\
\quad Nonwhite * percent Nonwhite            & $-0.97^{*}$   & $-1.75^{***}$ & $-1.37^{**}$  & $-0.93^{*}$  \\
                                                & $(0.42)$      & $(0.45)$      & $(0.47)$      & $(0.38)$     \\
\hline
\multicolumn{5}{l}{\scriptsize{$^{***}p<0.001$, $^{**}p<0.01$, $^*p<0.05$}}
\end{tabular}
\caption{Multinomial Logit Estimated Effects of Teacher Salary and Student Demographic Characteristics on the Probabilities That Teachers Switch School Districts or Exit Teaching Relative to Remaining in Same District}
\label{tbl:reg_mlogit}
\end{center}
\end{sidewaystable}

Finally, the conflation of switching districts and exiting teaching may
mask important heterogeneity between these two choices. To separate
these competing exit risks, HKR construct Table 9, which gives
coefficients from a multinomial logit model with three choices -- remain
in district, switch districts, and exit teaching. We repeat that
analysis here in Table \ref{tbl:reg_mlogit}, with the caveat that, given
the sparsity in racial variation present among Wisconsin teachers, we
are unable to identify the full model specified by HKR and mirrored
above in Tables \ref{tbl:reg_lpm} and \ref{tbl:reg_lpm_fe}. In light of
this, and in light of the apparent similarity in Wisconsin in the
behavior of black and Hispanic teachers described above, we specify the
multinomial logit model in terms of a more parsimonious coefficient set.
Namely, we distinguish between white and nonwhite teachers and white and
nonwhite students (instead of among white, black, and Hispanic students
and teachers).

We continue to see little evidence favoring the salience of wage
considerations for Wisconsin teachers; the strongest suggestions found
here point to the importance of wages for older male teachers in exiting
teaching, a result which is generally opposed to that found by HKR in
Texas, where salaries were generally important, but only for the
propensity to change districts. Also as in the regression
specifications, the prominence of student proficiency found by HKR fails
to make a notable appearance in Wisconsin.

With respect to the importance of student demographics, our results
again point to the same effects found in Texas. White teachers seem to
be spurred to change districts or exit teaching by highly black student
populations; the reverse is true of nonwhite teachers, who can be drawn
to remain in high-minority districts. Subsidized lunch eligibility's
strong effect observed in the combined specification is found here to be
concentrated more among those leaving teaching than those changing
districts.

\TAG{END_BODY}

\section{Conclusion}\label{conclusion}

That salary incentives appear to play such a limited role in driving
teacher churn is bound at first glance to be a disappointment for
policymakers. The most powerful predictors of turnover in educators in
Wisconsin are all basically beyond the control of administrators, who
have no readily-manipulated direct lever for assigning students to
schools\footnote{There is evidence (e.g., Richards
  \protect\hyperlink{ref-richards}{2014}) that catchment area
  manipulation (educational gerrymandering) is being used by some
  schools to select their student populations, but the equilibrium
  outcome of the strategic interactions of districts competing for the
  most ``desirable'' students is far from clear.}. HKR found school
quality (as measured by average standardized test performance) to be of
key importance for attracting/retaining teachers, but we found no
evidence that student proficiency (as measured by attainment levels on
standardized tests) is a factor in the turnover decision for Wisconsin
teachers. Regardless, manipulating school performance is famously
difficult\footnote{See, for example, the widespread cheating scandals on
  standardized tests by teachers in Chicago (Jacob and Levitt
  \protect\hyperlink{ref-jacob}{2003}), Atlanta, and Philadelphia as an
  example of the lengths professionals feel they need to go to effect
  change in testing outcomes.}, and is in fact the original goal
administrators often have in mind when they turn to labor market
policies in the first place, so that telling administrators they can
improve teacher retention by improving student performance amounts
essentially to circular reasoning.

The upside is that this paper is far from settling the debate about
welfare-maximizing teacher turnover policies. Limitations in our data
prevent us from associating to teachers anything but crude measures of
their productivity; measures such as experience, certification, and race
are famously poor predictors of teacher quality measures such as
value-added. We are thus unable to provide any input to the question of
whether \emph{high-quality} teachers have patterns of mobility which
resemble that of the teaching population as a whole, or whether
heterogeneity in their preferences can be used to devise appropriate
policies.

\section{Appendix: Longitudinal Teacher Panel from Unlinked
Cross-Sectional
Cuts}\label{appendix-longitudinal-teacher-panel-from-unlinked-cross-sectional-cuts}

\TAG{BEGIN_APPENDIX}

DPI WISEstaff data does not include a longitudinal identifier for
teachers, so we resorted to an alternative approach to matching teachers
from year to year. The essence of this algorithm relies on the inclusion
of four fields in the DPI data -- \texttt{first\_name},
\texttt{last\_name}, \texttt{nee} (former last name) and
\texttt{birth\_year}. By matching teacher using these identifiers, it is
possible to construct with high accuracy a panel of teachers from simple
teacher cross-sections\footnote{The code for this process was done using
  R and especially helped by the \texttt{data.table} package (Dowle and
  Srinivasan \protect\hyperlink{ref-dowle}{2017}). The code to reproduce
  this entire project can be found at
  \url{https://github.com/MichaelChirico/wisconsin_teachers}. The script
  for the algorithm is \texttt{teacher\_match\_and\_clean.R}. The
  \texttt{README} file gives steps for full reproduction, including
  retrieving the raw data.}.

\subsection{Step 0: Pre-Processing}\label{step-0-pre-processing}

Prior to beginning the matching process, a number of steps are taken to
improve the quality of the raw data. The first is to incorporate as many
of the errata mentioned in Public Instruction
(\protect\hyperlink{ref-dpi_errata}{2017}\protect\hyperlink{ref-dpi_errata}{a})
as possible. All name variables are then converted to lower case, after
which we extract maiden names (identified for those missing a
DPI-supplied entry in \texttt{nee} as the part of the
\texttt{last\_name} field that appears in between parentheses or
surrounding a hyphen or forward slash). Generally, it appears the maiden
name comes in the data \emph{before} the hyphen, but we create the
\texttt{nee2} field to identify potential matches to the post-hyphen
name as well. A search was done of the data for irregular characters
(punctuation or numbers) which allowed several obvious typos to be
resolved (e.g., \texttt{l0is\ a\ dewey} was easy to resolve as being
\texttt{lois\ a\ dewey}), and this is implemented next.

We then create a ``clean'' version of the name fields which strips away
all whitespace, initials (lone letters), and punctuation. At this stage,
all observations which identically match another in the same year from
the viewpoint of the algorithm -- namely, those that match exactly
another observation on the cleaned first and last names and year of
birth -- are removed from the data since it would be impossible to tell
such teachers apart. A more ambitious treatment would attempt to use
other cues found in the duplicated records (ethnicity, subject/position
cues, etc.) to separate such teachers, but the marginal cost of doing so
was found to exceed the potential benefit considerably for the exercise
at hand (recall that only 0.7\% of total observations are lost in this
fashion). Finally, teachers in the first year of data (1994-95) are
assigned an ID starting from 1 using the within-year identifier provided
by DPI.

\subsection{Steps 1-21: Matching}\label{steps-1-21-matching}

The algorithm proceeds by iterating over years of the data. In each year
\(Y\), matches are found serially by progressively adjusting the
criteria for considering two observations to be from the same teacher as
follows:

\begin{enumerate}
\def\labelenumi{\arabic{enumi}.}
\tightlist
\item
  Match anyone who stayed in the same school -- i.e., match any teacher
  found in a year \(Y'<Y\) with the same \texttt{first\_name},
  \texttt{last\_name}, and \texttt{birth\_year} at the same
  \texttt{district} and \texttt{school}.
\item
  Find within-district switchers -- those who match on all but the
  \texttt{school} field from Step 1.
\item
  Find out-of-district switchers -- those who match on all but the
  \texttt{district} field from Step 2.
\item
  Find teachers that appear to have been married, but have not moved --
  their \texttt{nee} field in \(Y\) is matched to the
  \texttt{last\_name} fields in \(Y'<Y\), but otherwise the fields from
  Step 1 are all matched. We create a flag for teachers matched in this
  fashion called \texttt{married}.
\item
  Repeat Step 2 for those who appear to have married.
\item
  Repeat Step 3 for those who appear to have married. 7-9. Repeat Steps
  4-6 using the \texttt{nee2} field instead of \texttt{nee}. 10-18.
  Repeat Steps 1-9 using the ``cleaned'' version of the first name field
  that had non-alphabetic characters removed,
  \texttt{first\_name\_clean}. We create a flag for teachers matched in
  this fashion called \texttt{mismatch\_inits}.
\item
  Match individuals in the same school assigned to the same position
  (identified in \texttt{position\_code}) but with different years of
  birth to overcome potential noise in year of birth (most commonly, the
  year of birth is missing in some years). We create a flag for teachers
  matched in this fashion called \texttt{mismatch\_yob}.
\item
  Match individuals in the same \emph{district} assigned to the same
  position but with different years of birth. We do not extend this
  logic to find district switchers since the potential for erroneous
  match assignment in such a case is too great, and we neglect to extend
  the algorithm to use other cues from the data to facilitate matching
  in such cases.
\item
  Assign new IDs to all teachers in \(Y\) not matched in the first 20
  steps, incrementing from the highest ID recorded thus far.
\end{enumerate}

To help ensure we are matching to the most important observation of each
teacher, matching is always done to a teacher's highest-FTE observation
within a year (particularly important for Steps 19-20). Further, it is
sometimes the case that a given tuple of search keys matches more than
one teacher in the prior data; if so, these rows are simply ignored for
that step and such a teacher will go unmatched unless they are uniquely
pinned down in a subsequent step.

\begin{figure}[htbp]
\centering
\includegraphics{figures/match_step-1.pdf}
\caption{\label{fig:step}Frequency of Matching by Step}
\end{figure}

Figure \ref{fig:step} Shows that teachers are most commonly matched in
the first three steps, meaning data fidelity issues are not \emph{per
se} devastating. The real benefits of the algorithm are in the
subsequent steps, as a result of which an additional 33,538 teachers are
matched than would have been if only the original first and last name
fields as found in the raw data were used.

\subsection{Step 22: Post-Processing}\label{step-22-post-processing}

The panels implied by the matched IDs created by the algorithm still
have a substantial amount of data quality issues which we can only
address once teachers' multiple observations are associated, mainly
having to do with instability in certain observable characteristics
which should be constant over time. First, we cascade forward maiden
names (if a teacher has non-missing \texttt{nee} in a period \(Y\) and
it becomes missing in \(Y'>Y\), we replace it in \(Y'\) with its value
in \(Y\)); the same is done for the certification field,
\texttt{highest\_degree} (just as a teacher cannot erase marriage from
their past, so can they not make a degree disappear).

Next, we correct instability in the \texttt{ethnicity} (and
\texttt{gender}) fields when possible according to three steps: 1) it is
sometimes missing, in which case we simply overwrite it with the other
values found for that teacher; 2) at least 70\% of a teacher's
observations use the same ethnicity (or gender); or 3) there are at
least five people that share a last name with an ethnicity-ambiguous
teacher, at least 70\% of whom have one ethnicity (or gender), the idea
being that names like Xu or Gutierrez are strongly associated with a
particular ethnicity. This type of correction is uncommon enough not to
warrant an appeal to a more sophisticated approach commonly found in
natural language processing applications, e.g.~training a classifier
such as a random forest (Breiman \protect\hyperlink{ref-breiman}{2001})
to predict ethnicity as accurately as possible.

Lastly, we synergize the year of birth field for those matched in Steps
19 or 20 by assigning the one that appears most frequently for each
teacher; in the case of ties, we use a regression-to-the-mean-type logic
and assign the year which brings the teacher closer to the median age
observed in the data. More data-driven approaches (conditioning the
target median on the teacher's employer, position, year in the data, or
using social security data to determine the maximum-likelihood year of
birth for a given first name, etc.) were again eschewed for expediency.

\subsection{\texorpdfstring{Validity Check:
\texttt{file\_number}}{Validity Check: file\_number}}\label{validity-check-file_number}

Starting in the 2011-12 release, the DPI data begins to consistently
record a field called \texttt{file\_number} for teachers which generally
acts as a time-consistent ID (from verbiage gleaned from Public
Instruction
\protect\hyperlink{ref-dpi_errata}{2017}\protect\hyperlink{ref-dpi_errata}{a},
it appears this corresponds to a teaching license number). We looked for
instances of multiple file numbers and are content that the algorithm is
performing well -- only 78 teacher IDs were found to be associated with
more than one \texttt{file\_number}, with almost all of them having been
matched on Steps 1 - 3 (what should be the highest-accuracy steps).
Given a number of apparent transcription mistakes (i.e.,
\texttt{file\_number} differing by one digit in some years) and that the
\texttt{file\_number} does appear to change on occasion, even these 78
could be an overstatement of the number of incorrectly matched
individuals.

\TAG{END_APPENDIX}

\section*{References}\label{references}
\addcontentsline{toc}{section}{References}

\hypertarget{refs}{}
\hypertarget{ref-anzia}{}
Anzia, Sarah F, and Terry M Moe. 2014. ``Collective Bargaining, Transfer
Rights, and Disadvantaged Schools.'' \emph{Educational Evaluation and
Policy Analysis} 36 (1). SAGE: 83--111.

\hypertarget{ref-ballou}{}
Ballou, Dale, and Michael Podgursky. 2002. ``Returns to Seniority Among
Public School Teachers.'' \emph{Journal of Human Resources}. University
of Wisconsin Press, 892--912.

\hypertarget{ref-biasi}{}
Biasi, Barbara. 2017. ``Unions, Salaries, and the Market for Teachers:
Evidence from Wisconsin.'' SSRN.
doi:\href{https://doi.org/http://dx.doi.org/10.2139/ssrn.2942134}{http://dx.doi.org/10.2139/ssrn.2942134}.

\hypertarget{ref-boyd2009}{}
Boyd, Donald J, Pamela L Grossman, Hamilton Lankford, Susanna Loeb, and
James Wyckoff. 2009. ``Teacher Preparation and Student Achievement.''
\emph{Educational Evaluation and Policy Analysis} 31 (4). SAGE
Publications Sage CA: Los Angeles, CA: 416--40.

\hypertarget{ref-boyd2005}{}
Boyd, Donald, Hamilton Lankford, Susanna Loeb, and James Wyckoff. 2005.
``Explaining the Short Careers of High-Achieving Teachers in Schools
with Low-Performing Students.'' \emph{The American Economic Review} 95
(2). American Economic Association: 166--71.

\hypertarget{ref-breiman}{}
Breiman, Leo. 2001. ``Random Forests.'' \emph{Machine Learning} 45 (1).
Springer: 5--32.

\hypertarget{ref-chettyI}{}
Chetty, Raj, John N Friedman, and Jonah E Rockoff. 2014a. ``Measuring
the Impacts of Teachers I: Evaluating Bias in Teacher Value-Added
Estimates.'' \emph{The American Economic Review} 104 (9). American
Economic Association: 2593--2632.

\hypertarget{ref-chettyII}{}
---------. 2014b. ``Measuring the Impacts of Teachers II: Teacher
Value-Added and Student Outcomes in Adulthood.'' \emph{The American
Economic Review} 104 (9). American Economic Association: 2633--79.

\hypertarget{ref-cohenvogel}{}
Cohen-Vogel, Lora, Li Feng, and La'Tara Osborne-Lampkin. 2013.
``Seniority Provisions in Collective Bargaining Agreements and the
`Teacher Quality Gap'.'' \emph{Educational Evaluation and Policy
Analysis} 35 (3). SAGE: 324--43.

\hypertarget{ref-croissant}{}
Croissant, Yves. 2012. ``Estimation of Multinomial Logit Models in R:
The Mlogit Package.'' \emph{R Package Version 0.2-2}.
\url{http://cran.r-project.org/web/packages/mlogit/vignettes/mlogit.pdf}.

\hypertarget{ref-dahl}{}
Dahl, David B. 2009. ``Xtable: Export Tables to LaTeX or HTML.'' \emph{R
Package Version 1.8.2}, 1--5.

\hypertarget{ref-deboor}{}
de Boor, Carl. 1978. \emph{A Practical Guide to Splines}. Vol. 27.
Springer-Verlag New York.

\hypertarget{ref-dolton}{}
Dolton, Peter, and Wilbert Van der Klaauw. 1999. ``The Turnover of
Teachers: A Competing Risks Explanation.'' \emph{Review of Economics and
Statistics} 81 (3). MIT Press: 543--50.

\hypertarget{ref-dowle}{}
Dowle, Matt, and Arun Srinivasan. 2017. \emph{Data.table: Extension of
`Data.frame`}. \url{http://r-datatable.com}.

\hypertarget{ref-engel}{}
Engel, Mimi, Brian A Jacob, and F Chris Curran. 2014. ``New Evidence on
Teacher Labor Supply.'' \emph{American Educational Research Journal} 51
(1). SAGE: 36--72.

\hypertarget{ref-fischer}{}
Fischer, Thomas A, Wayne H Swanger, and Stacey Skoning. 2009. ``Supply
\& Demand 2008: Data Trends of Education Personnel in Wisconsin Public
Schools.'' Wisconsin Educator Supply; Demand Project.

\hypertarget{ref-friedman}{}
Friedman, Jerome, Trevor Hastie, and Robert Tibshirani. 2001. \emph{The
Elements of Statistical Learning}. Vol. 1. Springer.

\hypertarget{ref-fulbeck}{}
Fulbeck, Eleanor S. 2014. ``Teacher Mobility and Financial Incentives: A
Descriptive Analysis of Denver's Procomp.'' \emph{Educational Evaluation
and Policy Analysis} 36 (1). SAGE: 67--82.

\hypertarget{ref-glazerman}{}
Glazerman, Steven, Ali Protik, Bing-Ru Teh, Julie Bruch, and Jeffrey
Max. 2013. ``Transfer Incentives for High-Performing Teachers: Final
Results from a Multisite Randomized Experiment.'' \emph{National Center
for Education Evaluation and Regional Assistance}. ERIC.

\hypertarget{ref-goldhaber2007}{}
Goldhaber, Dan, Betheny Gross, and Daniel Player. 2007. ``Are Public
Schools Really Losing Their Best? Assessing the Career Transitions of
Teachers and Their Implications for the Quality of the Teacher
Workforce.'' \emph{National Center for Analysis of Longitudinal Data in
Education Research}. CALDER.

\hypertarget{ref-goldhaber2015}{}
Goldhaber, Dan, Lesley Lavery, and Roddy Theobald. 2015. ``Uneven
Playing Field? Assessing the Teacher Quality Gap Between Advantaged and
Disadvantaged Students.'' \emph{Educational Researcher} 44 (5). SAGE:
293--307.

\hypertarget{ref-green}{}
Green, Peter J, and Bernard W Silverman. 1993. \emph{Nonparametric
Regression and Generalized Linear Models: A Roughness Penalty Approach}.
CRC Press.

\hypertarget{ref-hanushek2010}{}
Hanushek, Eric A, and Steven G Rivkin. 2010. ``Constrained Job Matching:
Does Teacher Job Search Harm Disadvantaged Urban Schools?'' National
Bureau of Economic Research.

\hypertarget{ref-hanushek}{}
Hanushek, Eric A, John F Kain, and Steven G Rivkin. 2004. ``Why Public
Schools Lose Teachers.'' \emph{Journal of Human Resources} 39 (2).
University of Wisconsin Press: 326--54.

\hypertarget{ref-harris}{}
Harris, Douglas N, and Tim R Sass. 2011. ``Teacher Training, Teacher
Quality and Student Achievement.'' \emph{Journal of Public Economics} 95
(7). Elsevier: 798--812.

\hypertarget{ref-he}{}
He, Xuming, and Pin Ng. 1999. ``COBS: Qualitatively Constrained
Smoothing via Linear Programming.'' \emph{Computational Statistics} 14
(3). Springer: 315--38.

\hypertarget{ref-heckman}{}
Heckman, James J, Lance J Lochner, and Petra E Todd. 2003. ``Fifty Years
of Mincer Earnings Regressions.'' National Bureau of Economic Research.

\hypertarget{ref-henningsen}{}
Henningsen, Arne, and Ott Toomet. 2011. ``MaxLik: A Package for Maximum
Likelihood Estimation in R.'' \emph{Computational Statistics} 26 (3):
443--58.
doi:\href{https://doi.org/10.1007/s00180-010-0217-1}{10.1007/s00180-010-0217-1}.

\hypertarget{ref-jacob}{}
Jacob, Brian A, and Steven D Levitt. 2003. ``Rotten Apples: An
Investigation of the Prevalence and Predictors of Teacher Cheating.''
\emph{The Quarterly Journal of Economics} 118 (3). Oxford University
Press: 843--77.

\hypertarget{ref-koski}{}
Koski, William S, and Eileen L Horng. 2007. ``Facilitating the Teacher
Quality Gap? Collective Bargaining Agreements, Teacher Hiring and
Transfer Rules, and Teacher Assignment Among Schools in California.''
\emph{Education} 2 (3). MIT Press: 262--300.

\hypertarget{ref-leifeld}{}
Leifeld, Philip. 2013. ``texreg: Conversion of Statistical Model Output
in R to LaTeX and HTML Tables.'' \emph{Journal of Statistical Software}
55 (8): 1--24. \url{http://www.jstatsoft.org/v55/i08/}.

\hypertarget{ref-litten}{}
Litten, Andrew. 2016. ``The Effects of Public Unions on Compensation:
Evidence from Wisconsin.''
\url{https://drive.google.com/file/d/0BwL-PvOgW_Ordzk5QXloZGM1d1k/view}.

\hypertarget{ref-loeb}{}
Loeb, Susanna, and Marianne E Page. 2000. ``Examining the Link Between
Teacher Wages and Student Outcomes: The Importance of Alternative Labor
Market Opportunities and Non-Pecuniary Variation.'' \emph{Review of
Economics and Statistics} 82 (3). MIT Press: 393--408.

\hypertarget{ref-moe}{}
Moe, Terry M. 2006. ``Bottom-up Structure: Collective Bargaining,
Transfer Rights, and the Plight of Disadvantaged Schools.''
\emph{Education Working Paper Archive}. ERIC.

\hypertarget{ref-murnane}{}
Murnane, Richard J, and Randall J Olsen. 1990. ``The Effects of Salaries
and Opportunity Costs on Length of Stay in Teaching: Evidence from North
Carolina.'' \emph{Journal of Human Resources}. University of Wisconsin
Press, 106--24.

\hypertarget{ref-ng}{}
Ng, Pin, and Martin Maechler. 2007. ``A Fast and Efficient
Implementation of Qualitatively Constrained Quantile Smoothing
Splines.'' \emph{Statistical Modelling} 7 (4). SAGE: 315--28.

\hypertarget{ref-dpi_name_change}{}
Public Instruction, Wisconsin Department of. 2011. ``School District
Name Changes - Since 1994.''
\url{dpi.wi.gov/sites/default/files/imce/sms/doc/rg_sdnamechanges.doc}.

\hypertarget{ref-dpi_errata}{}
---------. 2017a. ``Data Errata - Data Changes After Dpi Publication.''
\url{https://dpi.wi.gov/cst/data-collections/data-errata}.

\hypertarget{ref-dpi}{}
---------. 2017b. ``School Staff: Salary, Position \& Demographic
Reports.''
\url{https://dpi.wi.gov/cst/data-collections/staff/published-data}.

\hypertarget{ref-r}{}
R Core Team. 2016. \emph{R: A Language and Environment for Statistical
Computing}. Vienna, Austria: R Foundation for Statistical Computing.
\url{https://www.R-project.org/}.

\hypertarget{ref-richards}{}
Richards, Meredith P. 2014. ``The Gerrymandering of School Attendance
Zones and the Segregation of Public Schools: A Geospatial Analysis.''
\emph{American Educational Research Journal} 51 (6). SAGE: 1119--57.

\hypertarget{ref-rivkin}{}
Rivkin, Steven G, Eric A Hanushek, and John F Kain. 2005. ``Teachers,
Schools, and Academic Achievement.'' \emph{Econometrica} 73 (2). Journal
of the Econometric Society: 417--58.

\hypertarget{ref-rockoff}{}
Rockoff, Jonah E. 2004. ``The Impact of Individual Teachers on Student
Achievement: Evidence from Panel Data.'' \emph{The American Economic
Review} 94 (2). JSTOR: 247--52.

\hypertarget{ref-rstudio}{}
RStudio Team. 2017. \emph{RStudio: Integrated Development Environment
for R}. Boston, MA: RStudio, Inc. \url{http://www.rstudio.com/}.

\hypertarget{ref-runge}{}
Runge, Carl. 1901. ``Über Empirische Funktionen Und Die Interpolation
Zwischen Äquidistanten Ordinaten.'' \emph{Zeitschrift Für Mathematik Und
Physik} 46 (224-243): 20.

\hypertarget{ref-sable}{}
Sable, Jennifer. 2009. ``Documentation to the Nces Common Core of Data
Local Education Agency Universe Survey: School Year 2006-07 (Nces
2009-301).'' \emph{U.S. Department of Education}. Washington, DC:
National Center for Education Statistics.

\hypertarget{ref-scafidi}{}
Scafidi, Benjamin, David L Sjoquist, and Todd R Stinebrickner. 2007.
``Race, Poverty, and Teacher Mobility.'' \emph{Economics of Education
Review} 26 (2). Elsevier: 145--59.

\hypertarget{ref-stinebrickner}{}
Stinebrickner, Todd R. 2002. ``An Analysis of Occupational Change and
Departure from the Labor Force: Evidence of the Reasons That Teachers
Leave.'' \emph{Journal of Human Resources}. University of Wisconsin
Press, 192--216.

\hypertarget{ref-stone}{}
Stone, Mervyn. 1974. ``Cross-Validatory Choice and Assessment of
Statistical Predictions.'' \emph{Journal of the Royal Statistical
Society. Series B (Methodological)}. JSTOR, 111--47.

\hypertarget{ref-wahba}{}
Wahba, Grace. 1990. \emph{Spline Models for Observational Data}. SIAM.

\hypertarget{ref-wayne}{}
Wayne, Andrew J, Kwang Suk Yoon, Pei Zhu, Stephanie Cronen, and Michael
S Garet. 2008. ``Experimenting with Teacher Professional Development:
Motives and Methods.'' \emph{Educational Researcher} 37 (8). SAGE:
469--79.

\hypertarget{ref-xie}{}
Xie, Yihui. 2016. \emph{Knitr: A General-Purpose Package for Dynamic
Report Generation in R}. \url{http://yihui.name/knitr/}.

\hypertarget{ref-zeileis2004}{}
Zeileis, Achim. 2004. ``Econometric Computing with Hc and Hac Covariance
Matrix Estimators.'' \emph{Journal of Statistical Software} 11 (10):
1--17. \url{http://www.jstatsoft.org/v11/i10/}.

\hypertarget{ref-zeileis2006}{}
---------. 2006. ``Object-Oriented Computation of Sandwich Estimators.''
\emph{Journal of Statistical Software} 16 (9): 1--16.
\url{http://www.jstatsoft.org/v16/i09/.}

\hypertarget{ref-zeileis2002}{}
Zeileis, Achim, and Torsten Hothorn. 2002. ``Diagnostic Checking in
Regression Relationships.'' \emph{R News} 2 (3): 7--10.
\url{https://CRAN.R-project.org/doc/Rnews/}.


\end{document}
