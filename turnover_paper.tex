\documentclass[]{article}
\usepackage{lmodern}
\usepackage{amssymb,amsmath}
\usepackage{ifxetex,ifluatex}
\usepackage{fixltx2e} % provides \textsubscript
\ifnum 0\ifxetex 1\fi\ifluatex 1\fi=0 % if pdftex
  \usepackage[T1]{fontenc}
  \usepackage[utf8]{inputenc}
\else % if luatex or xelatex
  \ifxetex
    \usepackage{mathspec}
  \else
    \usepackage{fontspec}
  \fi
  \defaultfontfeatures{Ligatures=TeX,Scale=MatchLowercase}
\fi
% use upquote if available, for straight quotes in verbatim environments
\IfFileExists{upquote.sty}{\usepackage{upquote}}{}
% use microtype if available
\IfFileExists{microtype.sty}{%
\usepackage{microtype}
\UseMicrotypeSet[protrusion]{basicmath} % disable protrusion for tt fonts
}{}
\usepackage[margin=1in]{geometry}
\usepackage{hyperref}
\hypersetup{unicode=true,
            pdftitle={Teacher Turnover in Wisconsin},
            pdfauthor={Michael Chirico},
            pdfborder={0 0 0},
            breaklinks=true}
\urlstyle{same}  % don't use monospace font for urls
\usepackage{graphicx,grffile}
\makeatletter
\def\maxwidth{\ifdim\Gin@nat@width>\linewidth\linewidth\else\Gin@nat@width\fi}
\def\maxheight{\ifdim\Gin@nat@height>\textheight\textheight\else\Gin@nat@height\fi}
\makeatother
% Scale images if necessary, so that they will not overflow the page
% margins by default, and it is still possible to overwrite the defaults
% using explicit options in \includegraphics[width, height, ...]{}
\setkeys{Gin}{width=\maxwidth,height=\maxheight,keepaspectratio}
\IfFileExists{parskip.sty}{%
\usepackage{parskip}
}{% else
\setlength{\parindent}{0pt}
\setlength{\parskip}{6pt plus 2pt minus 1pt}
}
\setlength{\emergencystretch}{3em}  % prevent overfull lines
\providecommand{\tightlist}{%
  \setlength{\itemsep}{0pt}\setlength{\parskip}{0pt}}
\setcounter{secnumdepth}{0}
% Redefines (sub)paragraphs to behave more like sections
\ifx\paragraph\undefined\else
\let\oldparagraph\paragraph
\renewcommand{\paragraph}[1]{\oldparagraph{#1}\mbox{}}
\fi
\ifx\subparagraph\undefined\else
\let\oldsubparagraph\subparagraph
\renewcommand{\subparagraph}[1]{\oldsubparagraph{#1}\mbox{}}
\fi

%%% Use protect on footnotes to avoid problems with footnotes in titles
\let\rmarkdownfootnote\footnote%
\def\footnote{\protect\rmarkdownfootnote}

%%% Change title format to be more compact
\usepackage{titling}

% Create subtitle command for use in maketitle
\newcommand{\subtitle}[1]{
  \posttitle{
    \begin{center}\large#1\end{center}
    }
}

\setlength{\droptitle}{-2em}
  \title{Teacher Turnover in Wisconsin}
  \pretitle{\vspace{\droptitle}\centering\huge}
  \posttitle{\par}
  \author{Michael Chirico}
  \preauthor{\centering\large\emph}
  \postauthor{\par}
  \predate{\centering\large\emph}
  \postdate{\par}
  \date{March 17, 2017}

\usepackage{theapa}
\usepackage{array}
\usepackage{multirow}
\newcolumntype{R}[1]{>{\raggedleft\arraybackslash}p{#1\linewidth}}
\usepackage{rotating}

\begin{document}
\maketitle
\begin{abstract}
Given the consistently-affirmed importance of teacher quality to student
success, understanding teacher churn is crucial to formulating and
evaluating teacher labor market policy. This paper replicates the
analysis of Hanushek, Kain, and Rivkin
(\protect\hyperlink{ref-hanushek}{2004}) over a longer and more recent
time period in Wisconsin and confirms all of its major findings, namely
that while pay is a inter-district pay differentials are a significant
determinant of turnover, school quality measures are much better
predictors of all three types of churn -- within and between school
districts and out of local public schools.
\end{abstract}

\section{Introduction}\label{introduction}

\section{Literature Review}\label{literature-review}

Because the potential policy implications of turnover in the teaching
profession (from human capital and equity/distributional perspectives
both) are far-reaching and polypartisan, the literature on
turnover-related topics in education is extensive. As relates to this
paper, there are five broad (and often overlapping) categories of
inquiry: the relationship between turnover and wages, which has tended
to focus on ``opportunity wages'' outside of the field of education; the
relationship between turnover, school demographics, and other
nonpecuniary benefits, which has tended to focus on distributional
inequalities--whether teachers with certain characteristics are more or
less likely to be teaching certain disadvantaged groups; the
relationship between turnover and teacher quality as measured by student
performance, usually value added; collective bargaining agreements in
education, focusing by and large on the implications (or lack thereof)
of seniority-preferential clauses; and the recent phenomenon of specific
retention incentives, the provisioning of wage bonuses to teachers
willing to teach in high-needs schools.

One of the earliest papers attempting to rigorously investigate turnover
was a panel study of teachers in Michigan by Murnane and Olsen
(\protect\hyperlink{ref-murnane}{1990}), who used college degree field
wages outside of education as opportunity wages, finding the expected
lower exit rate for teachers with higher wages in teaching relative to
the authors' defined alternative. Dolton and Van der Klaauw
(\protect\hyperlink{ref-dolton}{1999}) use panel data on university
graduates in the United Kingdom to estimate a competing risks model of
the decision to leave teaching entirely, finding results in line with
Murnane and Olsen (\protect\hyperlink{ref-murnane}{1990}). Returning to
panel studies in the US, Loeb and Page
(\protect\hyperlink{ref-loeb}{2000}) use PUMS data to get an idea of
teacher relative wages in many states and find that dropout rates fall
when teacher relative wages are high. Stinebrickner
(\protect\hyperlink{ref-stinebrickner}{2002}) also uses panel data (this
time NLS-72) to track both teachers and non-teachers, focusing in
particular on young teachers who leave the profession for long stints,
and finds that the best predictor of female exit is recent childbearing,
which is an important consideration for all work related to teacher
turnover because such a high percentage (76 nationwide) of teachers are
female. Lastly, Hanushek, Kain, and Rivkin
(\protect\hyperlink{ref-hanushek}{2004}) focuses on teachers in Texas
and emphasizes that the characteristics of students are much stronger
factors in predicting teacher exit than are wages (while also affirming
the statistical significance of pay).

While wages have been found consistently to have some measurable effect
on teacher turnover, it is impossible to explain within-district
migration (which constitutes a large portion of switching--as much as
50\%) through wage-only channels because contracts are fixed at the
district level. As such, another strand of literature has chosen to
focus on the nonpecuniary aspects of the decision to take a teaching
job--school environment/rapport, student enthusiasm, neighborhood
characteristics, etc.--usually by directing attention to a single
district so that any wage-based considerations are stifled, as is the
case for Boyd et al. (\protect\hyperlink{ref-boyd}{2005}) and Engel,
Jacob, and Curran (\protect\hyperlink{ref-engel}{2014}). Boyd et al.
(\protect\hyperlink{ref-boyd}{2005}) track early-career teachers in New
York City as they quit or transfer out of the city, and most importantly
finds that commuting time is an important, often overlooked aspect of
location preference. Engel, Jacob, and Curran
(\protect\hyperlink{ref-engel}{2014}) leverages a unique data set from
Chicago Public School job fairs which affords them a rather strong
measure of teachers' demand for vacancies, neutralizing the influence of
school administration's behavior on turnover (through poor match
selection or other means). The authors contribute evidence that the
school's neighborhood (perhaps due to ambient crime or other
reputational effects good and bad) is a better predictor of teachers'
preference than distance from home, going somewhat against the grain of
Boyd et al. (\protect\hyperlink{ref-boyd}{2005}). Scafidi, Sjoquist, and
Stinebrickner (\protect\hyperlink{ref-scafidi}{2007}) examine statewide
data from Georgia, but ignore wage effects, choosing instead to focus on
disentangling the contributions of low student achievement and minority
status to turnover; they find that minority status is the more salient
associate of teacher exit.

The key element missing from all of the above studies is perhaps the
most important consideration in the issue of teacher turnover--teacher
quality. None of the studies above have student-teacher matched data,
and so are unable to directly associate student outcomes with any given
teacher. If, with respect to any measure of quality you would like, we
find that transitioning teachers are identical to their replacements,
the issue of teacher turnover is not, in fact, much of an issue--it
leans closer to hot air and wasted ink. Thus, the recent trend in the
literature to incorporate measures of teacher quality (in large part
made possible by a trend towards administrative records allowing
students to be linked to teachers and tracked over time) in
considerations of teacher turnover has made big strides in addressing
the most policy-relevant questions to be asked. The most common and
widely accepted measure of teacher quality is value added\footnote{The
  most commonly cited expositions on value added, its validity, and so
  on are probably Rivkin, Hanushek, and Kain
  (\protect\hyperlink{ref-rivkin}{2005}), an extensive exploration of
  the predictive powers of empirical Bayes VA measures; and Chetty,
  Friedman, and Rockoff
  (\protect\hyperlink{ref-chettyI}{2014}\protect\hyperlink{ref-chettyI}{a})
  and Chetty, Friedman, and Rockoff
  (\protect\hyperlink{ref-chettyII}{2014}\protect\hyperlink{ref-chettyII}{b}),
  the largest-scale study of long-term inferences based on value added.}
(in its various guises), and the literature has begun to incorporate
such measures into studies of teacher turnover. Hanushek and Rivkin
(\protect\hyperlink{ref-hanushek2010}{2010}) considers value added as a
measure of teacher productivity, and ask if common results of labor
search theory (namely that turnover falls with tenure and that turnover
is negatively associated with match-specific productivity) continue to
hold in the education labor market. In fact, the authors find that the
teachers most likely to switch schools are those with low measured match
quality, and especially that those who leave teaching entirely are those
with the lowest match quality. The results are more pronounced for
schools with high proportions of low-SES students, which has strong
policy implications, as it appears the best teachers in high needs
schools are the least likely to change jobs. Goldhaber, Gross, and
Player (\protect\hyperlink{ref-goldhaber2007}{2007}) performs a similar
analysis with the longitudinal data of North Carolina and comes to
similar conclusions, strengthening the robustness of the results.
Lastly, Goldhaber, Lavery, and Theobald
(\protect\hyperlink{ref-goldhaber2015}{2015}) examine the inequity in
the distribution of teacher quality by high-needs groups in Washington
state, and find that for all three measures of quality (teacher
experience, licensure exam score, and value added), the distribution of
teachers favors the less needy (as measured by free/reduced-price lunch
status, minority status, and low prior academic achievement).

The aforementioned papers have tended to keep the collective bargaining
aspect of salary determination for teachers out of the spotlight, if
largely for reasons of data restrictions. Nevertheless, it stands to
reason to believe that the rigid structure of union-negotiated contracts
could serve to contribute in a large way to teacher turnover. Ballou and
Podgursky (\protect\hyperlink{ref-ballou}{2002}) give much descriptive
evidence of the shape of the wage-tenure profile, rooted in a data set
collected by the Department of Defense and published by the AFT. They
find that seniority premia in education largely mirror those in more
traditional white collar professins, that steeper profiles are
associated with less turnover, and that district financial and
demographic conditions alone are insufficient to explain variation in
contracts. Another common (and recently quite controversial, as
evidenced by the contention in the ongoing contract negotiations in
Philadelphia) feature of union-negotiated teacher contracts are
seniority priviliges--preferential treatments granted to teachers in
voluntary and involuntary transfers. Moe
(\protect\hyperlink{ref-moe}{2006}) codes contracts from 158 districts
in California according to the strength of seniority rights therein
guaranteed to teachers and finds that such rights are associated with
the distribution of teachers across schools (measuring quality as
experience and certification) in a way that serves to harm minorities.
Revisiting California with a slightly different sample and definition of
the ``determinacy'' of the contracts with respect to seniority, Koski
and Horng (\protect\hyperlink{ref-koski}{2007}) come to the opposite
conclusion--that there is no such relationship. As a rebuttal, Anzia and
Moe (\protect\hyperlink{ref-anzia}{2014}) pin the difference in results
on the exclusion in Moe (\protect\hyperlink{ref-moe}{2006}) of small
school districts, where it appears that the entrenchment of bureaucracy
falters and the rigidity of contract language wane, a claim which they
support by repeating their analysis with the inclusion of an interaction
for district size--indeed, for small districts the result of Koski and
Horng (\protect\hyperlink{ref-koski}{2007}) holds, while the insight of
Moe (\protect\hyperlink{ref-moe}{2006}) holds in larger districts.
Cohen-Vogel, Feng, and Osborne-Lampkin
(\protect\hyperlink{ref-cohenvogel}{2013}) use data from Florida and
their results align with those of Koski and Horng
(\protect\hyperlink{ref-koski}{2007}) (though they neglect to nuance
their results by district size).

Finally, an emerging but still immature strand of literature is
beginning to look at the potential for transfer bonuses and retention
incentives to positively affect student outcomes. Fulbeck
(\protect\hyperlink{ref-fulbeck}{2014}) analyzes a scheme in place in
Denver whereby teachers who choose to transfer to high-needs schools
(low-performing) are given recurring bonus pay, and those initially
stationed there are given retention incentives. She concludes that
recipients of incentives are significantly less likely to switch jobs,
as driven by a reduction in district exit rates and especially by
teachers whos incentive payments exceed \$5,000. Glazerman et al.
(\protect\hyperlink{ref-glazerman}{2013}) evaluate the Talent Transfer
Initiative, a randomized controlled trial conducted in 10 districts
whereby high-performance teachers were given \$20,000 over the course of
two years as reward for transferring the identified high-needs schools,
and conclude that there were significant effects on teacher retention as
well as on student outcomes.

\section{Data}\label{data}

The State of Wisconsin's Department of Public Instruction (DPI) releases
annual Salary, Position \& Demographic reports through the WISEstaff
data collection system, and these reports represent ``a point-in-time
collection of all staff members in public schools as of the 3rd Friday
of September\ldots{}'' (Public Instruction
\protect\hyperlink{ref-dpi}{2017}), which serve as the primary source of
data on teachers in this paper. Data are available at the
position-teacher level cross-sectionally, with each entry corresponding
to one of possibly several positions held by each school district
employee. Identifiers in each file permit unique identification of an
employee within a given year, but this identifier does not follow
teachers between years. To overcome this substantial hurdle to
identifying teacher mobility, data are first fed through the matching
algorithm described in further detail in the appendix. Essentially, we
are aided by the presence of various imperfect identifiers which are
more stable over time, most crucially teachers' first and last names and
years of birth. By building on these covariates and incorporating some
limited fuzzy matching techniques, we construct a panel of teachers
spanning the 1994-95 academic year (AY) through AY2015-16.\footnote{For
  brevity, we herein refer to academic years by the spring year, e.g.,
  AY2003-04 will be simply 2004.}

As noted in the companion paper, the introduction of Wisconsin Act 10
introduced a substantial structural break in the labor market for
Wisconsin teachers, so we include only data from 2000-2008 to avoid
conflating the effects of this policy on teacher turnover, a topic
covered in more detail in the companion paper and elsewhere, with the
earlier functioning of the labor market (i.e., we do not want to mix the
results from distinct equilibria of the teacher labor market, but would
prefer to analyze the pre- and post-Act-10 markets separately). We drop
all employees who are not full-time, full-year regular teachers of a
major core subject (all-purpose elementary teachers or English/Math) at
a single regular public school with a Bachelor's or Master's degree and
fewer than 50 years' recorded experience; taken together, these
restrictions eliminate 77\% of employees, the lion's share of which come
from eliminating substitutes/support staff and teachers non-core
subjects. We then eliminate teachers with missing information on their
subsequent school or district and teachers with instability in their
recorded ethnicity, as well as teachers not categorized as white, black,
or Hispanic, eliminating a further 1\% of all employees\footnote{Wisconsin
  teachers are predominantly white (96\%).}. Finally, we drop teachers'
multiple positions by keeping only the highest-intensity position for
each teacher, as measured by full-time equivalency, resulting in a final
count of 253,935 teacher-year observations.

This data is also used for the incorporation of counterfactual salary
calculations, by incorporating the salary schedules estimated in the
companion paper. Details of the fit procedure can be found there, but
essentially salary schedules are computed as monotonicity- and
concavity-constrained median-targeted splines (Ng and Maechler
\protect\hyperlink{ref-ng}{2007}) for each level of certification
(Bachelor's or Master's degree) in each district in each year\footnote{One
  difference is that the salaries included in the payscales estimation
  were less restrictive with respect to included subject areas. This was
  done since contracts are collectively bargained at the district level
  for all teachers, with scant mention of subject area in wage
  determination.}. Data sparsity led this procedure to be unreliable in
many cases, so ultimately around 29\% of teachers have missing salary
information\footnote{More specifically, we eliminate district-years
  featuring less than 20 teachers, less than 7 distinct levels of
  observed experience, or less than 5 unique values of the two measures
  of pay (salary and fringe benefits) in either degree track. HKR
  include like-minded restrictions, but combine teachers of different
  certification within an experience level.}, mostly in rural districts
or other districts with only one or two schools and a small number of
students.

We supplement the DPI teacher salary data set in several ways to
incorporate data about other characteristics of schools and districts in
Wisconsin. To get school- and district-level measures of socioeconomic
makeup (percentage of students who are black or Hispanic or eligible for
free/reduced lunches) and community type/urbanicity, we tap the National
Center for Education Statistics' Common Core of Data's Universe Surveys,
which provide this information on a yearly basis for all years in the
study\footnote{The method of recording urbanicity by the Common Core
  switched from being ``metropolitan-centric'' to being
  ``urban-centric'' for Wisconsin from 2006 (Sable
  \protect\hyperlink{ref-sable}{2009}). We map the codes corresponding
  codes to match those used by HKR as well as possible, and use the data
  file from 2006, which has both types of code for all US districts, to
  confirm that this correspondence is by and large working as intended.
  For a small number of districts/schools with missing urbanicity codes
  in certain years, we use information from other years to inform
  urbanicity.}. At the district level, we also use this data to compute
class size and the size of the student body.

Lastly, we turn to DPI's public data again to get school- and
district-level performance metrics. While Hanushek, Kain, and Rivkin
(\protect\hyperlink{ref-hanushek}{2004}) were able to obtain school- and
district-level average scale scores on a standardized test in Texas,
such a metric is not publicly available in Wisconsin for all years.
Instead, we calculate student proficiency rates for each school and
district as the percentage of test-takers deemed to be at grade level in
mathematics or reading in a given year on the Wisconsin Knowledge and
Concepts Examination (WKCE), which is administered to 4th, 8th, and
10th-grade students.

\section{Results}\label{results}

Table \ref{tbl:move_by_exp} replicates Table 1 of Hanushek, Kain, and
Rivkin (\protect\hyperlink{ref-hanushek}{2004}) (HKR), and as HKR found
in Texas, most turnover in Wisconsin is happening within districts and
out of the profession. In Wisconsin, the fraction of teachers
transitioning among districts is vanishingly small after a ``burn-in''
period of roughly 6 years -- only 0.8 of such teachers do so (compared
with 3.1\% for the comparable group in HKR), but is still relatively
higher among the youngest teachers -- roughly twice as high for the
``probationary'' teachers (1-3 years' experience) as for teachers with
7-11 years' experience in both states.

By contrast, movement patterns within districts in the two states are
very similar, lending weight to teachers ``earning their stripes''
within a district to be able to choose the best schools as a privilege
of seniority. As expected, we also observe a U-shaped pattern in
teachers exiting Wisconsin public schools, which jives with two types of
quits. Early-career quitters who change to private schools, change state
of residence, or change professions, and late-career quitters who
retire. Results not included here break down the exit rates by
experience level, where this dichotomy is even more dramatic --
first-year exit rates are about 9 percent and quickly level off at
around 2 percent before spiking again past around 25 years.

As examined further below, the low rate of switches between districts
appears to be owing to the generally more rural nature of Wisconsin
vis-`\{a\}-vis. Texas. To wit, Milwaukee is the only major urban area in
the state, and its population (2010 Census) of 594,833 would rank 7th in
Texas. This means that two major types of movers in the HKR data --
Large Urban - Large Urban and Suburban - Large Urban -- are limited
within the state to ending up in a relatively minor metropolitan area.
HKR don't provide any results disaggregated by city, precluding any
attempts to compare these numbers more comparably to those that would
obtain from eliminating the largest cities in Texas.

\begin{table}[ht]
\centering
\begin{tabular}{p{.103\linewidth}p{.14\linewidth}p{.15\linewidth}p{.09\linewidth}p{.16\linewidth}R{.16}}
  \hline
 & \multicolumn{4}{c}{Percent of Teachers Who} & \\ \cline{2-5}
Teacher Experience & Remain in Same School & Change Schools Within District & Switch Districts & Exit Wisconsin \mbox{Public Schools} & Number of Teachers \\ 
  \hline
1-3 years & 85.2 & 9.4 & 5.4 & 7.3 & 37,044 \\ 
  4-6 years & 88.9 & 8.0 & 3.1 & 4.6 & 33,972 \\ 
  7-11 years & 91.1 & 7.2 & 1.7 & 2.8 & 48,047 \\ 
  12-30 years & 94.2 & 5.3 & 0.5 & 3.0 & 113,334 \\ 
  >30 years & 96.7 & 3.0 & 0.4 & 15.1 & 21,538 \\ 
  All & 91.8 & 6.4 & 1.8 & 4.8 & 253,935 \\ 
   \hline
\end{tabular}
\caption{Year-to-year Transitions of Teachers by Experience, 2000-08} 
\label{tbl:move_by_exp}
\end{table}

Table \ref{tbl:markov} replicates HKR Table 2, and supports its most
important conclusions. HKR argue that there is little support for the
idea that scores of young teachers are using large urban schools as a
training ground before ``settling down'' with easier assignments in the
suburb, based on the general low level of turnover from Large Urban
districts. We affirm the scarcity of transitions from districts in
Milwaukee, while also noting that such a path is certainly present, as
evidenced by the majority of those who do leave Large Urban districts
ending up in a Surburban district in both settings.

As mentioned in the discussion of Table \ref{tbl:move_by_exp}, the major
difference with respect to quantities observed in Texas appears to be
driven in differences in the urban landscape between Texas and
Wisconsin\footnote{We also note a difference in the relative shift in
  population between the two states -- Texas observed dramatic changes
  in its community type distribution over the period of study of only 4
  years, while Wisconsin only saw some movement from Rural to Suburban
  communities.}. This is supported by the overall similarity of
magnitudes of transition rates to community types besides Large Urban in
the two papers. Again, the ``stickiest'' community type is Rural -- over
60\% of Rural teachers remain Rural in both papers, and even fewer Rural
Wisconsin teachers end up in a big city than is the case for Texas. Also
as in HKR, we find broad similarity in the community type transition
patterns of younger teachers as compared to all teachers.

\begin{sidewaystable}[ht]
\centering
\begin{tabular}{lrrrrrrr}
  \hline
 & \multicolumn{4}{c}{\multirow{2}{*}{Percent of Teachers Who Move to}} & \multirow{4}{*}{\parbox{0.09\linewidth}{Number Teachers Changing Districts}} & \multirow{4}{*}{\parbox{0.07\linewidth}{Percent of Origin Teachers}} & \multirow{4}{*}{\parbox{0.07\linewidth}{Change in Share of Teachers 2000-06}}\\
 & \multicolumn{4}{c}{} & & & \\ \cline{2-5}
& & & & & & & \\
Origin Community & Large Urban & Small Urban & Suburban & Rural &  &  &  \\ 
  \hline
I. All teachers & & & & & & & \\
\quad Large Urban & 5.8 & 14.6 & 58.9 & 20.6 & 459 & 1.8 & -0.3\% \\ 
  \quad Small Urban & 3.3 & 13.1 & 45.2 & 38.4 & 500 & 1.1 & -0.2\% \\ 
  \quad Suburban & 3.7 & 15.2 & 45.0 & 36.1 & 1,210 & 1.7 & 4.1\% \\ 
  \quad Rural & 0.8 & 11.4 & 24.3 & 63.5 & 2,377 & 2.1 & -3.5\% \\ 
\multicolumn{3}{l}{II. Probationary teachers (1-3 years experience)} & & & & & \\
  \quad Large Urban & 7.8 & 15.9 & 56.3 & 20.0 & 260 & 3.5 &  \\ 
  \quad Small Urban & 4.4 & 12.4 & 46.9 & 36.3 & 230 & 3.5 &  \\ 
  \quad Suburban & 4.5 & 16.1 & 41.6 & 37.7 & 495 & 5.2 &  \\ 
  \quad Rural & 0.5 & 11.4 & 25.4 & 62.6 & 1,024 & 7.6 &  \\ 
   \hline
\end{tabular}
\caption{Destination Community Type for Teachers Changing Districts, by Origin Community Type and Teacher Experience Level} 
\label{tbl:markov}
\end{sidewaystable}

Table \ref{tbl:change_by_ge} replicates Table 3 of HKR, and confirms its
most important insights. Raw salary differentials predict teacher
mobility, but the pay differential is not on average very large -- only
about \$200, or 0.4\% higher than the counterfactually expected wage
that would have obtained had the district-switching teacher remained in
their current district. This premium declines with age for both male and
female teachers, eventually dipping negative (though this estimate is
imprecise/underpowered due to the limited quantity of teachers changing
districts after 6 years).

Attempting to isolate the influence of district characteristics on wage
effects, HKR suggest comparing the differential leverage of residual
wages to get a more focused estimate of the association between wages
and mobility. We run a similar regression using the payscales estimated
in the companion paper, but evaluate separate regressions not just for
each level of experience, but also for each certification track. This
leads to a boost in the overall fraction of explained variance from 60\%
cited by HKR to NA\% here; as in HKR, other included covariates are
consistently significant, suggesting their strong independent
correlation with salary levels.

As in HKR, we find the demographic-independent wage differentials to be
even more important than the uncontrolled raw wages, with the predicted
wage improvement nearly doubling to 0.8\%. In contrast to HKR, however,
we find a positive relationship between experience and residual wage
differentials, with mid-career district switchers experiencing 2-3\%
higher wages upon arrival to their new employer, by contrast to the null
relationship for probationary teachers.

Student demographic differentials are very important for predicting
teacher turnover, a finding which held in Texas as it does in Wisconsin.
Most important in all experience classes and for both genders are the
measures of student performance and student poverty -- district
switchers end up at schools with 4\% more students at grade level
overall, an effect which is stronger for female teachers and for young
teachers. They also end up on average with about 7\% fewer students
(school-wide) eligible for subsidized lunch. While this finding would
need to be bolstered with experimental or quasi-experimental evidence,
it hints at the potentially limited scope of teacher labor market
policies intended to ameliorate teacher supply problems in hard-to-serve
districts -- schools can much more easily exert influence over their
compensation policies than they can dictate their student bodies, but
the latter is more efficacious (see Fulbeck
\protect\hyperlink{ref-fulbeck}{2014} and Glazerman et al.
(\protect\hyperlink{ref-glazerman}{2013})).

\begin{sidewaystable}[ht]
\centering
\begin{tabular}{lccccccc}
  \hline
 & \multicolumn{3}{c}{Men by Experience Class} & \multicolumn{3}{c}{Women by Experience Class} & \multirow{2}{*}{\parbox{0.09\linewidth}{All Teachers 0-9 Years}}\\ \cline{2-4} \cline{5-7}
 & 1-3 years & 4-6 years & 7-11 years & 1-3 years & 4-6 years & 7-11 years &  \\ 
  \hline
Base year salary (log) & 0.004 & 0.022 & -0.021 & 0.010 & 0.002 & -0.016 & 0.004 \\ 
   & (0.010) & (0.015) & (0.022) & (0.005) & (0.009) & (0.012) & (0.004) \\ 
  Adjusted salary (log) & -0.012 & 0.007 & 0.031 & 0.001 & 0.018 & 0.023 & 0.007 \\ 
   & (0.007) & (0.011) & (0.016) & (0.004) & (0.007) & (0.009) & (0.003) \\ 
  Percent proficient & 2.9\% & 1.9\% & 1.6\% & 4.8\% & 3.9\% & 4.1\% & 3.9\% \\ 
   & (0.7\%) & (0.8\%) & (1.1\%) & (0.4\%) & (0.5\%) & (0.6\%) & (0.2\%) \\ 
  Percent Hispanic & -1.6\% & -0.3\% & -0.6\% & -1.7\% & -1.7\% & -1.1\% & -1.5\% \\ 
   & (0.3\%) & (0.4\%) & (0.5\%) & (0.2\%) & (0.2\%) & (0.3\%) & (0.1\%) \\ 
  Percent black & -3.4\% & -1.1\% & -3.2\% & -5.1\% & -3.3\% & -4.8\% & -4.1\% \\ 
   & (0.9\%) & (1.0\%) & (1.1\%) & (0.5\%) & (0.7\%) & (0.8\%) & (0.3\%) \\ 
  Percent subsidized lunch & -6.9\% & -3.8\% & -3.9\% & -8.8\% & -6.1\% & -5.9\% & -7.0\% \\ 
   & (1.1\%) & (1.4\%) & (1.6\%) & (0.6\%) & (0.8\%) & (1.0\%) & (0.4\%) \\ 
   \hline
\end{tabular}
\caption{Average Change in Salary and District Student Characteristics (and Standard Deviations) for Teachers Changing Districts, by Gender and Experience} 
\label{tbl:change_by_ge}
\end{sidewaystable}

\begin{table}[ht]
\centering
\begin{tabular}{lp{.12\textwidth}p{.12\textwidth}p{.12\textwidth}p{.12\textwidth}}
  \hline
 & \multicolumn{2}{c}{District Average Characteristics} & \multicolumn{2}{c}{Campus Average Characteristics}\\ \cline{2-5}
 & Large Urban to Suburban & Suburban to Suburban & Large Urban to Suburban & Suburban to Suburban \\
  \hline
Base year salary (log) & -0.014 & 0.019 & --- & --- \\ 
   & (0.013) & (0.008) &  &  \\ 
  Adjusted salary (log) & -0.037 & 0.015 & --- & --- \\ 
   & (0.011) & (0.006) &  &  \\ 
Average Student Characteristics & & & & \\
  \quad Percent proficient & 37.1\% & 0.7\% & 33.4\% & 0.1\% \\ 
   & (0.5\%) & (0.4\%) & (1.3\%) & (0.6\%) \\ 
  \quad Percent Hispanic & -13.2\% & -0.6\% & -8.3\% & -0.8\% \\ 
   & (0.3\%) & (0.2\%) & (1.3\%) & (0.3\%) \\ 
  \quad Percent black & -52.8\% & -0.4\% & -56.6\% & -0.5\% \\ 
   & (0.5\%) & (0.3\%) & (1.9\%) & (0.4\%) \\ 
  \quad Percent subsidized lunch & -60.7\% & -1.5\% & -61.8\% & -2.0\% \\ 
   & (0.7\%) & (0.5\%) & (1.2\%) & (0.6\%) \\ 
   \hline
\end{tabular}
\caption{Average Change in Salary and in District and Campus Student Characteristics (and Standard Deviations) for Teachers with 1-10 Years of Experience Who Change Districts, by Community Type of Origin and Destination District} 
\label{tbl:change_by_urb}
\end{table}

\begin{table}[ht]
\centering
\begin{tabular}{lp{.1\textwidth}p{.1\textwidth}p{.1\textwidth}p{.1\textwidth}}
  \hline
 & \multicolumn{2}{c}{Between District Moves} & \multicolumn{2}{c}{Within District Moves}\\ \cline{2-5}
 & Black Teachers & Hispanic Teachers & Black Teachers & Hispanic Teachers \\
  \hline
Percent proficient & 16.5\% & 5.1\% & 3.2\% & 1.7\% \\ 
   & (4.9\%) & (8.7\%) & (0.9\%) & (1.4\%) \\ 
  Percent Hispanic & -1.1\% & -14.2\% & 0.0\% & -7.2\% \\ 
   & (1.7\%) & (7.8\%) & (0.8\%) & (2.3\%) \\ 
  Percent black & -22.9\% & 1.7\% & -1.6\% & 0.3\% \\ 
   & (7.8\%) & (5.8\%) & (1.4\%) & (2.0\%) \\ 
  Percent subsidized lunch & -59.0\% & -12.7\% & -2.6\% & -3.7\% \\ 
   & (6.5\%) & (9.4\%) & (0.6\%) & (1.3\%) \\ 
  Number of teachers & 42 & 22 & 549 & 197 \\ 
   \hline
\end{tabular}
\caption{Average Change in District and Campus Student Characteristics (and Standard Deviations) for Black and Hispanic Teachers with 1-10 Years of Experience who Change Campuses} 
\label{tbl:change_by_eth}
\end{table}

\begin{table}[ht]
\centering
\begin{tabular}{p{.25\textwidth}p{.13\textwidth}p{.13\textwidth}p{.13\textwidth}}
  \hline
Quartile of Distribution & Probability Teachers Move to New School within District & Probability Teachers Move to New District & Probability Teachers Exit Public Schools \\ 
  \hline
Residual salary & & & \\
\quad Highest & --- & 1.3\% & 4.8\% \\ 
  \quad 3rd & --- & 1.4\% & 5.1\% \\ 
  \quad 2nd & --- & 1.6\% & 5.0\% \\ 
  \quad Lowest & --- & 1.9\% & 5.4\% \\ 
Percent proficient & & & \\
  \quad Highest & 5.6\% & 1.7\% & 4.7\% \\ 
  \quad 3rd & 6.8\% & 1.9\% & 4.4\% \\ 
  \quad 2nd & 6.2\% & 1.9\% & 5.4\% \\ 
  \quad Lowest & 6.9\% & 1.7\% & 4.8\% \\ 
Percent eligible for reduced-price lunch & & & \\
  \quad Highest & 7.7\% & 1.7\% & 5.6\% \\ 
  \quad 3rd & 6.9\% & 1.6\% & 4.3\% \\ 
  \quad 2nd & 6.2\% & 1.9\% & 4.5\% \\ 
  \quad Lowest & 4.9\% & 2.0\% & 4.8\% \\ 
Percent Black & & & \\
  \quad Highest & 6.7\% & 1.6\% & 6.2\% \\ 
  \quad 3rd & 5.7\% & 1.5\% & 4.7\% \\ 
  \quad 2nd & 6.6\% & 1.8\% & 4.5\% \\ 
  \quad Lowest & 6.7\% & 2.2\% & 3.9\% \\ 
Percent Hispanic & & & \\
  \quad Highest & 6.9\% & 1.5\% & 6.0\% \\ 
  \quad 3rd & 5.6\% & 1.9\% & 4.8\% \\ 
  \quad 2nd & 6.0\% & 1.8\% & 4.5\% \\ 
  \quad Lowest & 7.1\% & 1.9\% & 4.1\% \\ 
   \hline
\end{tabular}
\caption{School Average Transition Rates by Distribution of Residual Teacher Salary and Student Demographic Characteristics (data weighted by number of teachers in school)} 
\label{tbl:change_by_quartile}
\end{table}

\begin{table}
\begin{center}
\begin{tabular}{l c c c c c }
\hline
 & \multicolumn{4}{c}{Teacher Experience} \\ \cline{2-6}
 & 1-3 years & 4-6 years & 7-11 years & 12-30 years & >30 years \\
\hline
First year base salary (log)                & $-0.04$      & $-0.05$       & $-0.07^{*}$   & $0.00$       & $-0.16^{*}$  \\
                                            & $(0.05)$     & $(0.04)$      & $(0.03)$      & $(0.02)$     & $(0.07)$     \\
First year base salary (log) * female       & $-0.08$      & $0.05$        & $0.05$        & $-0.02$      & $0.10$       \\
                                            & $(0.05)$     & $(0.04)$      & $(0.03)$      & $(0.02)$     & $(0.08)$     \\
Campus average student characteristics      &              &               &               &              &              \\
\quad Percent proficient                    & $-0.06^{**}$ & $0.02$        & $-0.01$       & $-0.01$      & $-0.01$      \\
                                            & $(0.02)$     & $(0.02)$      & $(0.01)$      & $(0.01)$     & $(0.03)$     \\
\quad Percent eligible for subsidized lunch & $-0.03^{*}$  & $-0.07^{***}$ & $-0.04^{***}$ & $-0.02^{**}$ & $0.10^{***}$ \\
                                            & $(0.02)$     & $(0.01)$      & $(0.01)$      & $(0.01)$     & $(0.03)$     \\
\quad Percent Black                         & $0.05^{**}$  & $0.11^{***}$  & $0.04^{***}$  & $0.05^{***}$ & $0.06$       \\
                                            & $(0.02)$     & $(0.02)$      & $(0.01)$      & $(0.01)$     & $(0.04)$     \\
\quad Percent Hispanic                      & $0.03$       & $0.09^{***}$  & $0.04^{**}$   & $0.01$       & $-0.05$      \\
                                            & $(0.02)$     & $(0.02)$      & $(0.01)$      & $(0.01)$     & $(0.05)$     \\
Interactions                                &              &               &               &              &              \\
\quad Black * percent Black                 & $-0.09$      & $-0.10^{**}$  & $-0.01$       & $-0.00$      & $-0.23^{**}$ \\
                                            & $(0.05)$     & $(0.03)$      & $(0.02)$      & $(0.02)$     & $(0.08)$     \\
\quad Hispanic * percent Black              & $-0.15^{*}$  & $-0.18^{**}$  & $-0.13^{**}$  & $-0.08$      & $-0.22$      \\
                                            & $(0.06)$     & $(0.06)$      & $(0.05)$      & $(0.05)$     & $(0.32)$     \\
\quad Black * percent Hispanic              & $-0.05$      & $-0.03$       & $-0.00$       & $0.03$       & $0.14$       \\
                                            & $(0.09)$     & $(0.06)$      & $(0.04)$      & $(0.04)$     & $(0.22)$     \\
\quad Hispanic * percent Hispanic           & $-0.12^{*}$  & $-0.16^{***}$ & $-0.13^{***}$ & $-0.08^{**}$ & $-0.49$      \\
                                            & $(0.05)$     & $(0.05)$      & $(0.03)$      & $(0.03)$     & $(0.38)$     \\
\hline
Observations                                & 28,287        & 25,609         & 35,946         & 81,801        & 14,773        \\
\hline
\multicolumn{6}{l}{\scriptsize{$^{***}p<0.001$, $^{**}p<0.01$, $^*p<0.05$}}
\end{tabular}
\caption{Estimated Effects of Starting Teacher Salary and Student Demographic Characteristics on the Probability that Teachers Leave School Districts, by Experience (linear probability models; standard errors in parentheses)}
\label{tbl:reg_lpm}
\end{center}
\end{table}

\begin{table}
\begin{center}
\begin{tabular}{l c c c c c }
\hline
 & \multicolumn{4}{c}{Teacher Experience} \\ \cline{2-6}
 & 1-3 years & 12-30 years & 4-6 years & 7-11 years & >30 years \\
\hline
First year base salary (log)                & $-0.09$       & $-0.02$      & $-0.10^{*}$  & $-0.12^{***}$ & $-0.31^{***}$ \\
                                            & $(0.05)$      & $(0.02)$     & $(0.04)$     & $(0.03)$      & $(0.09)$      \\
First year base salary (log) * female       & $-0.08$       & $-0.02$      & $0.05$       & $0.05$        & $0.11$        \\
                                            & $(0.05)$      & $(0.02)$     & $(0.04)$     & $(0.03)$      & $(0.08)$      \\
Campus average student characteristics      &               &              &              &               &               \\
\quad Percent proficient                    & $-0.08^{***}$ & $-0.02^{**}$ & $0.00$       & $-0.02^{*}$   & $-0.03$       \\
                                            & $(0.02)$      & $(0.01)$     & $(0.02)$     & $(0.01)$      & $(0.03)$      \\
\quad Percent eligible for subsidized lunch & $-0.01$       & $-0.00$      & $-0.03$      & $-0.01$       & $0.05$        \\
                                            & $(0.02)$      & $(0.01)$     & $(0.02)$     & $(0.01)$      & $(0.04)$      \\
\quad Percent Black                         & $0.02$        & $0.03^{***}$ & $0.06^{**}$  & $0.00$        & $0.09$        \\
                                            & $(0.02)$      & $(0.01)$     & $(0.02)$     & $(0.01)$      & $(0.05)$      \\
\quad Percent Hispanic                      & $-0.01$       & $-0.00$      & $0.03$       & $-0.01$       & $0.01$        \\
                                            & $(0.03)$      & $(0.01)$     & $(0.02)$     & $(0.02)$      & $(0.06)$      \\
Interactions                                &               &              &              &               &               \\
\quad Black * percent Black                 & $-0.08$       & $0.01$       & $-0.09^{**}$ & $0.00$        & $-0.23^{**}$  \\
                                            & $(0.05)$      & $(0.02)$     & $(0.03)$     & $(0.02)$      & $(0.08)$      \\
\quad Hispanic * percent Black              & $-0.14^{*}$   & $-0.08$      & $-0.18^{**}$ & $-0.11^{*}$   & $-0.33$       \\
                                            & $(0.06)$      & $(0.05)$     & $(0.06)$     & $(0.05)$      & $(0.32)$      \\
\quad Black * percent Hispanic              & $-0.03$       & $0.04$       & $-0.01$      & $0.01$        & $0.16$        \\
                                            & $(0.09)$      & $(0.04)$     & $(0.06)$     & $(0.04)$      & $(0.22)$      \\
\quad Hispanic * percent Hispanic           & $-0.11^{*}$   & $-0.08^{**}$ & $-0.15^{**}$ & $-0.10^{**}$  & $-0.65$       \\
                                            & $(0.05)$      & $(0.03)$     & $(0.05)$     & $(0.04)$      & $(0.38)$      \\
\hline
Observations                                & 28,287         & 81,801        & 25,609        & 35,946         & 14,773         \\
\hline
\multicolumn{6}{l}{\scriptsize{$^{***}p<0.001$, $^{**}p<0.01$, $^*p<0.05$}}
\end{tabular}
\caption{Estimated Effects of Starting Teacher Salary and Student Demographic Characteristics on the Probability that Teachers Leave School Districts with District Fixed Effects, by Experience (linear probability models; standard errors in parentheses)}
\label{tbl:reg_lpm_fe}
\end{center}
\end{table}

\section{Conclusion}\label{conclusion}

\section*{References}\label{references}
\addcontentsline{toc}{section}{References}

\hypertarget{refs}{}
\hypertarget{ref-anzia}{}
Anzia, Sarah F, and Terry M Moe. 2014. ``Collective Bargaining, Transfer
Rights, and Disadvantaged Schools.'' \emph{Educational Evaluation and
Policy Analysis} 36 (1). SAGE: 83--111.

\hypertarget{ref-ballou}{}
Ballou, Dale, and Michael Podgursky. 2002. ``Returns to Seniority Among
Public School Teachers.'' \emph{Journal of Human Resources}. University
of Wisconsin Press, 892--912.

\hypertarget{ref-boyd}{}
Boyd, Donald, Hamilton Lankford, Susanna Loeb, and James Wyckoff. 2005.
``Explaining the Short Careers of High-Achieving Teachers in Schools
with Low-Performing Students.'' \emph{The American Economic Review} 95
(2). American Economic Association: 166--71.

\hypertarget{ref-chettyI}{}
Chetty, Raj, John N Friedman, and Jonah E Rockoff. 2014a. ``Measuring
the Impacts of Teachers I: Evaluating Bias in Teacher Value-Added
Estimates.'' \emph{The American Economic Review} 104 (9). American
Economic Association: 2593--2632.

\hypertarget{ref-chettyII}{}
---------. 2014b. ``Measuring the Impacts of Teachers II: Teacher
Value-Added and Student Outcomes in Adulthood.'' \emph{The American
Economic Review} 104 (9). American Economic Association: 2633--79.

\hypertarget{ref-cohenvogel}{}
Cohen-Vogel, Lora, Li Feng, and La'Tara Osborne-Lampkin. 2013.
``Seniority Provisions in Collective Bargaining Agreements and the
`Teacher Quality Gap'.'' \emph{Educational Evaluation and Policy
Analysis} 35 (3). SAGE: 324--43.

\hypertarget{ref-dolton}{}
Dolton, Peter, and Wilbert Van der Klaauw. 1999. ``The Turnover of
Teachers: A Competing Risks Explanation.'' \emph{Review of Economics and
Statistics} 81 (3). MIT Press: 543--50.

\hypertarget{ref-engel}{}
Engel, Mimi, Brian A Jacob, and F Chris Curran. 2014. ``New Evidence on
Teacher Labor Supply.'' \emph{American Educational Research Journal} 51
(1). SAGE: 36--72.

\hypertarget{ref-fulbeck}{}
Fulbeck, Eleanor S. 2014. ``Teacher Mobility and Financial Incentives: A
Descriptive Analysis of Denver's Procomp.'' \emph{Educational Evaluation
and Policy Analysis} 36 (1). SAGE: 67--82.

\hypertarget{ref-glazerman}{}
Glazerman, Steven, Ali Protik, Bing-Ru Teh, Julie Bruch, and Jeffrey
Max. 2013. ``Transfer Incentives for High-Performing Teachers: Final
Results from a Multisite Randomized Experiment.'' \emph{National Center
for Education Evaluation and Regional Assistance}. ERIC.

\hypertarget{ref-goldhaber2007}{}
Goldhaber, Dan, Betheny Gross, and Daniel Player. 2007. ``Are Public
Schools Really Losing Their Best? Assessing the Career Transitions of
Teachers and Their Implications for the Quality of the Teacher
Workforce.'' \emph{National Center for Analysis of Longitudinal Data in
Education Research}. CALDER.

\hypertarget{ref-goldhaber2015}{}
Goldhaber, Dan, Lesley Lavery, and Roddy Theobald. 2015. ``Uneven
Playing Field? Assessing the Teacher Quality Gap Between Advantaged and
Disadvantaged Students.'' \emph{Educational Researcher} 44 (5). SAGE:
293--307.

\hypertarget{ref-hanushek2010}{}
Hanushek, Eric A, and Steven G Rivkin. 2010. ``Constrained Job Matching:
Does Teacher Job Search Harm Disadvantaged Urban Schools?'' National
Bureau of Economic Research.

\hypertarget{ref-hanushek}{}
Hanushek, Eric A, John F Kain, and Steven G Rivkin. 2004. ``Why Public
Schools Lose Teachers.'' \emph{Journal of Human Resources} 39 (2).
University of Wisconsin Press: 326--54.

\hypertarget{ref-koski}{}
Koski, William S, and Eileen L Horng. 2007. ``Facilitating the Teacher
Quality Gap? Collective Bargaining Agreements, Teacher Hiring and
Transfer Rules, and Teacher Assignment Among Schools in California.''
\emph{Education} 2 (3). MIT Press: 262--300.

\hypertarget{ref-loeb}{}
Loeb, Susanna, and Marianne E Page. 2000. ``Examining the Link Between
Teacher Wages and Student Outcomes: The Importance of Alternative Labor
Market Opportunities and Non-Pecuniary Variation.'' \emph{Review of
Economics and Statistics} 82 (3). MIT Press: 393--408.

\hypertarget{ref-moe}{}
Moe, Terry M. 2006. ``Bottom-up Structure: Collective Bargaining,
Transfer Rights, and the Plight of Disadvantaged Schools.''
\emph{Education Working Paper Archive}. ERIC.

\hypertarget{ref-murnane}{}
Murnane, Richard J, and Randall J Olsen. 1990. ``The Effects of Salaries
and Opportunity Costs on Length of Stay in Teaching: Evidence from North
Carolina.'' \emph{Journal of Human Resources}. University of Wisconsin
Press, 106--24.

\hypertarget{ref-ng}{}
Ng, Pin, and Martin Maechler. 2007. ``A Fast and Efficient
Implementation of Qualitatively Constrained Quantile Smoothing
Splines.'' \emph{Statistical Modelling} 7 (4). SAGE: 315--28.

\hypertarget{ref-dpi}{}
Public Instruction, Wisconsin Department of. 2017. ``School Staff:
Salary, Position \& Demographic Reports.''
\url{https://dpi.wi.gov/cst/data-collections/staff/published-data}.

\hypertarget{ref-rivkin}{}
Rivkin, Steven G, Eric A Hanushek, and John F Kain. 2005. ``Teachers,
Schools, and Academic Achievement.'' \emph{Econometrica} 73 (2). Journal
of the Econometric Society: 417--58.

\hypertarget{ref-sable}{}
Sable, Jennifer. 2009. ``Documentation to the Nces Common Core of Data
Local Education Agency Universe Survey: School Year 2006-07 (Nces
2009-301).'' \emph{U.S. Department of Education}. Washington, DC:
National Center for Education Statistics.

\hypertarget{ref-scafidi}{}
Scafidi, Benjamin, David L Sjoquist, and Todd R Stinebrickner. 2007.
``Race, Poverty, and Teacher Mobility.'' \emph{Economics of Education
Review} 26 (2). Elsevier: 145--59.

\hypertarget{ref-stinebrickner}{}
Stinebrickner, Todd R. 2002. ``An Analysis of Occupational Change and
Departure from the Labor Force: Evidence of the Reasons That Teachers
Leave.'' \emph{Journal of Human Resources}. University of Wisconsin
Press, 192--216.


\end{document}
